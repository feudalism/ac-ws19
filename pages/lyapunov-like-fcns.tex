\paragraph{New interpretation of Lyapunov}
Nothing to do with energy.
$\bm{V}$ affects the scaling of the distance of $\bm{x}$ 
from the origin in the phase portrait.
\begin{align*}
\norm{\bm{x}}^2_{\bm{V}} = \bm{x}^T \bm{V} \bm{x}, \qquad \bm{V} \succ 0
\end{align*}~

\begin{conclusion}{darkgreen}
\textbf{All Lyapunov says is}: how far is $\bm{x}$ 
from the origin? We want to find some type of measure
for that.
\end{conclusion}

\paragraph{Lyapunov function} (Lyapunov-like)\\
We want the output error $e(t)$ as well as the parameter
error $\bm{ \tilde{\theta}}(t)$ to go to zero.
$\bm{\Gamma} \succ 0$ symmetrical, positive definite.
\begin{alignat*}{3}
V(e, \tilde{\bm{\theta}})
    &= \frac{1}{2}e^2 + \frac{1}{2}|k_p| \left( \tilde{\bm{\theta}}^T \bm{\Gamma}^{-1} \tilde{\bm{\theta}} \right)
    \numberthis \label{eq:V}\\
\dot{V}
    &= e \dot{e} + \frac{1}{2}|k_p|
    \undertxt{\left( 2 \tilde{\bm{\theta}}^T \bm{\Gamma}^{-1} \dot{\tilde{\bm{\theta}}} \right)}{\footnotemark}
\end{alignat*}
\footnotetext{Possible due to $\bm{\Gamma}$ symmetrical.}

Substitute $\dot{e}$ using equation \eqn{eq:error-dynamics-nonlinear}.
\begin{alignat*}{3}
\dot{V}  &= a_me^2 + e k_p \tilde{\bm{\theta}}^T \bm{\phi}
        + \frac{1}{2} |k_p| \left( 2 \tilde{\bm{\theta}^T \bm{\Gamma}^{-1} \dot{\tilde{\bm{\theta}}}} \right)\\
    &= a_me^2 + |k_p| \tilde{\bm{\theta}}^T
        \underbrace{\left(
        \sgn{k_p} e \phi + \bm{\Gamma}^{-1} \dot{\tilde{\theta}}\right)}_{
        \overset{!}{=} 0}
\end{alignat*}

The second term is set to 0, because we want $V \preceq 0$ and
we don't know all the signs of the terms.
This will define the \textbf{adaptive law}.
\begin{alignat*}{3}
\dot{\tilde{\theta}}(t) &= - \sgn{k_p} \bm{\Gamma} \bm{\phi}(t) e(t)\\
\dot{\theta}(t) &= - \sgn{k_p} \bm{\Gamma} \bm{\phi}(t) e(t)\\
\end{alignat*}

With the adaptive law, we obtain for $\dot{V}$:
\begin{alignat}{3}
\dot{V} &= a_m e^2 \preceq 0 \label{eq:Vdot}
\end{alignat}

\paragraph{Remark}
$e(t)$ does not have to be 0 -- why?
% \begin{itemize}
% \item We do not need $\dot{V}$ to approach zero.
% \item $\dot{V}=0$ does not imply that $V$ has a limit as $t \rightarrow \infty$.
%     (although this is known, see footnote below%
%     \footnote{A function $V$ that is bounded from below $V \succeq 0$
%         and \textbf{non-increasing} $\dot{V} \preceq 0$
%         has a limit as $t \rightarrow \infty$}).
% \item In other words, $\dot{V}=0$ does not imply that the errors go to zero,
%     and vice versa.
% \end{itemize}~  

\begin{conclusion}{darkred}
If the derivative of a function $\rightarrow 0$,
that \textcolor{red}{does not} imply that the function
has a limit, and vice versa:
if a function has a limit, that doesn't mean its
derivative $\rightarrow 0$.
\begin{alignat*}{3}
\lim_{t \rightarrow \infty} \dot{f}(t) = 0
    \nLeftrightarrow \lim_{t \rightarrow \infty} f(t) = k
\end{alignat*}
\end{conclusion}~

Counterexamples:
\begin{align*}
f(t) &= \sin \left( \ln t \right)\\
\nexists \lim_{t \rightarrow \infty} f(t), \quad
\dot{f}(t) &= \frac{\cos \left( \ln t \right)}{t} \rightarrow 0\\
~\\
f(t) &= e^{-t} \sin (e^{2t})\\
 \lim_{t \rightarrow \infty} f(t) &= 0\\
\dot{f}(t) &= -e^{-t}\sin(e^{2t}) + e^t \sin(e^{2t}) \\
    & \quad \rightarrow    \txt{ explodes!}
\end{align*}

\paragraph{Are the error dynamics stable?}
\begin{itemize}
\item Measure (some of) the states
\item Apply $V = f(e, \tilde{\bm{\theta}})$
\item $V \rightarrow \infty$? Or $V \downarrow$?\\
    $\Rightarrow$ analyse time derivative $\dot{V}$
\item If we show $\dot{V} \rightarrow 0$, then $e \rightarrow 0$.
\end{itemize}

\paragraph{Extensions to Lyapunov}
There are two well-known extensions to Lyapunov to prove
asymptotic stability, even if $\dot{V} \preceq 0$.
\begin{enumerate}
\item LaSalle's invariance principle
    (\redtext{only for autonomous systems})
\item Barbalat's lemma
    (\greentext{OK for non-autonomous systems})
\end{enumerate}~

Our system's error dynamics are non-autonomous,
$\dot{e} = f(t, \dots)$, due to
following another system.
\begin{figure}[H]
\centering
\inkscape[\normalsize]{e-non-auto}{0.25}
\end{figure}

