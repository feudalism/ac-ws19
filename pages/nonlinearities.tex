\begin{thrm}{Nonlinear SISO plant}
\paragraph{Given} plant $G$ and reference model $M$.
\begin{alignat*}{3}
G&:~ &
    \dot{x}_p(t) &= \unknownVar{a_p} x_p(t) + \unknownVar{k_p} u(t) \\
          &&& \quad \quad + \unknownVar{\alpha_p} f(\knownVar{z})
    \tag{\ref{eq:G-nonlinear}}\\
M&:~ &
    \dot{x}_m(t) &= \knownVar{a_m} x_m(t) + \knownVar{k_m} r(t)
    \tag{\ref{eq:M}}
\end{alignat*}

The input
\begin{alignat*}{3}
u(t) &= \bm{\theta}^T(t) \bm{\phi}(t) \tag{\ref{eq:u-nonlinear}}\\
    \tag{\ref{eq:adaptive-law}}
    \qquad \txt{with } \dot{ \bm{\theta}} &= -\sgn{k_p} \bm{\Gamma} \bm{\phi} e\\
    \qquad \txt{and } \bm{\phi}(t) &= \matr{r(t) & x_p(t) & f(z)}^T
\end{alignat*}
renders the origin asymptotically stable and guarantees
$x_p(t) \rightarrow x_m(t)$ as $t \rightarrow \infty$.
\end{thrm}~



We can add, arbitrarily, many `nonlinearities' $f_i(z_j)$ with
unknown gains $\alpha_i$ (Figure \ref{fig:nonlinearities}).
The function $z(t)$ $z$ is a placeholder. $z(t)$
can be an external or an internal signal.
The nonlinearity functions
\greentext{need not be continuous}.
The only requirement:\\

\begin{conclusion}{darkred}
    \begin{alignat*}{3}
        f_i(z_j) \in \mathcal{L}_\infty
    \end{alignat*}
    Nonlinearities are bounded at all times%
    \footnote{For the boundedness of $u(t)$ and therefore $\dot{e}(t)$}.
\end{conclusion}

\begin{figure}[H]
    \centering
    \tikzsetnextfilename{nonlinearities}
    {\footnotesize
\begin{tikzpicture}
\def\w{0.5cm}
\def\h{1cm}

\node (kp) [block] {$k_p$};
\node (comp1) [comp, right=\w of kp] {};
\node (comp2) [comp, right=\w of comp1] {};
\node (i) [block,right=1.5*\w of comp2] {$\int$} {};
\node (cout) [cout, right=1.5*\w of i] {};

\node (ap) [block] at ($($(comp1)!0.5!(cout)$) + (0,0.7*\h)$) {$a_p$};

\node (f2) [nonlin] at ($($(comp2)!0.7!(cout)$)+(0,-\h)$) {$f_2$};
\node (alp2) [block, left=0.8*\w of f2.west] {$\alpha_2$};

\node(alp1) [block, below=0.5*\h of comp1, anchor=north] {$\alpha_1$};
\node (f1) [nonlin, below=0.5*\h of alp1.south, anchor=north, text width=0.6cm] {$f_1(.)$};

\begin{scope}[arrow]
\draw ($(kp.west) + (-\w,0)$) -- (kp) node [pos=0.25,above] {$u$};
\draw (kp) -- (comp1);
\draw (comp1) -- (comp2);
\draw (comp2) -- (i) node [pos=0.5,above] {$ \dot{x}_p$};
\draw (i) -- (cout) -- ++(\w,0) node [pos=1,above] {$x_p$};

\draw (cout) |- (ap);
\draw (ap) -| (comp1);

\draw (cout) |- (f2);
\draw (f2) -- (alp2);
\draw (alp2) -| (comp2);

\draw ($(f1.south)+(0,-0.5*\h)$) -- (f1) node [pos=0.2,right] {$d$ (disturbances)};
\draw (f1) -- (alp1);
\draw (alp1) -- (comp1);
\end{scope}
\end{tikzpicture}
}

    \caption{Plant $G$ with nonlinearities}
    \label{fig:nonlinearities}
\end{figure}

\paragraph{Questions}
\begin{itemize}
\item Can we do $f(u)$?\\
    Possible, but \redtext{solving for $u = \cdots f(u) \cdots$ is difficult}.
\item Can $f(.)$ be a differential operator (filter)?\\
    \greentext{Yes} (Figure \ref{fig:nonlin-filter}).
    If the filter is linear, then a solution
    definitely exists.
    \begin{figure}[H]
        \centering
        \tikzsetnextfilename{nonlin-filter}
        {\footnotesize
\begin{tikzpicture}
\def\w{0.5cm}
\def\h{1cm}

\node (start) {};
\node (coutu) [cout, right=0.5*\w of start] {};
\node (comp1) [comp, right=5*\w of coutu] {};
\node (kp) [block] at ($(coutu)!0.5!(comp1)$) {$k_p$};
\node (cout) [cout, right=5*\w of comp1] {};
\node (i) [block] at ($(comp1)!0.5!(cout)$) {$\int$} {};
 
\node (ap) [block] at ($($(comp1)!0.5!(cout)$) + (0,0.7*\h)$) {$a_p$};
 
\node (fx) [nonlin] at ($($(comp1)!0.7!(cout)$)+(0,-\h)$) {$F_x$};
\node (filternote) [note,below=2mm of fx.south,anchor=center] {many filters};
\node (alp2) [block] at ($($(comp1)!0.3!(cout)$)+(0,-\h)$) {$\bm{\alpha}_x^T$};

\node (fu) [nonlin] at ($($(coutu)!0.3!(comp1)$)+(0,-\h)$) {$F_u$};
\node (alp1) [block] at ($($(coutu)!0.7!(comp1)$)+(0,-\h)$) {$\bm{\alpha}_p^T$};
 
\begin{scope}[arrow]
    \draw (start) -- (coutu)  node [pos=0.1,above] {$u$}
                  -- (kp) ;
    \draw (kp) -- (comp1);
    \draw (comp1) -- (i) node [pos=0.5,above] {$ \dot{x}_p$};
    \draw (i) -- (cout) -- ++(\w,0) node [pos=1,above] {$x_p$};
     
    \draw (cout) |- (ap);
    \draw (ap) -| (comp1);
     
    \draw (cout) |- (fx);
    \draw (alp2) -| (comp1);
     
    \draw (coutu) |- (fu);
    \draw (alp1.east) -- ++(2mm, 0) -- (comp1);
\end{scope}

\begin{scope}[vectline]
    \draw (fx) -- (alp2);
    \draw (fu) -- (alp1);
\end{scope}
\end{tikzpicture}
}

        \caption{Plant $G$ with nonlinearities as filters}
        \label{fig:nonlin-filter}
    \end{figure}
    The filters $F_u$ and  $F_x$ are stable dynamical systems.
    \redtext{If they are not stable, there is a higher chance of
    getting an infinite output $\rightarrow$ conflicts with bounded
    output requirement.}
\end{itemize}~

If no such filters are present in the plant $G$,
then the plant has an order of 1.\\

Let's say we have an \redtext{offset in the plant input} of unknown magnitude,
and the plant has otherwise \greentext{known parameters}.
\begin{alignat*}{3}
\dot{x}_p(t) &= \knownVar{a_p} x_p(t) + \knownVar{k_p} u(t) + \unknownVar{\alpha_p} \cdot 1\\
u(t) &= \knownVar{a^*} x_p(t) + \knownVar{k^*} r(t) - \frac{ \unknownVar{\alpha(t)}}{k_p}
\end{alignat*}

Closed loop becomes
\begin{alignat*}{3}
\dot{x}_p(t) &= a_m x_p(t) + k_m r(t) + \tilde{\alpha}(t)
\end{alignat*} 

Error dynamics
\begin{alignat*}{3}
\dot{e} &= a_m e(t) + \tilde{\alpha}(t)
\end{alignat*}

Lyapunov-like function
\begin{alignat*}{3}
V &= \frac{1}{2} e^2 + \frac{1}{2} \tilde{\alpha}^2\\
\dot{V} &= e \dot{e} + \tilde{\alpha} \dot{ \tilde{\alpha}} + \cdots\\
    &= a_me^2 + \tilde{\alpha} \undertxt{\left( e + \dot{ \tilde{\alpha}} \right)}{to set to 0}
\end{alignat*}

Setting the second term to zero ensures the negative
semi\-definite\-ness of $\dot{V}$.
\begin{alignat}{3}
\dot{ \tilde{\alpha}} &= -e \label{eq:integral-error}
\end{alignat}

Equation \eqn{eq:integral-error} above implies
$u = \cdots + \int e dt + \cdots$, i.e.,
that the controller contains an I-part.
As a result of this, there are no steady state errors
caused by model uncertainties.
The controller eliminates offset at input of plant.\\

Equation \eqn{eq:integral-error} is a pure integrator
acting on control error. This is a linear controller!
We have learned: integrators `learn' input offsets of
the plant and correct them.\\

\begin{conclusion}{darkgreen}
Adaptive controllers can be interpreted as nonlinear
PI controllers.
\end{conclusion}
