\section{Linear SISO plant}
\begin{problem}{Control problem}
\paragraph{Given} plant $G$ and reference model $M$.
\begin{alignat}{3}
    G&:~    & \dot{x}_p(t) &= a_p x_p(t) + k_p \unknownVar{u(t)},
            \label{eq:G-linear}\\
            &&& \txt{IC } x_p(0) \in \R \nonumber
\end{alignat}
\begin{variables}
    a_p     & pole of plant\\
    k_p     & input gain of plant\\
\end{variables}

\begin{alignat}{3}
    M&:~    & \dot{x}_m(t) &= \knownVar{a_m} x_m(t)
                        + \knownVar{k_m} \knownVar{r(t)},
            \label{eq:M}\\
            &&& \txt{IC } x_m(0) \in \R \nonumber
\end{alignat}
\begin{variables}
    \knownVar{a_m}     & pole of reference model\\
    \knownVar{k_m}     & input gain of reference model\\
    \knownVar{r(t)}    & reference signal
\end{variables}%
%
The \greentext{reference model parameters} are set by the user
and are therefore known.

\paragraph{Task} find a control $u(t)$ such that
$x_p(t) \rightarrow x_m(t)$ for $t\rightarrow \infty$.
\begin{alignat}{3}
    \tund{G}{des}:~
    && \dot{x}_p(t)    &= \knownVar{a_m} x_p(t)
                    + \knownVar{k_m r(t)}
                \label{eq:G-des}\\
                &&&\txt{IC } x_p(0) \in \R \nonumber
\end{alignat}
\end{problem}
~

Solutions for $u(t)$ using
\begin{itemize}
\item Model reference control (MRC)
\item Model reference adaptive control (MRAC)
\end{itemize}

\subsection{Model reference control (MRC)}
The \greentext{plant parameters} are assumed to be known.
\begin{alignat*}{3}
    G&:~    & \dot{x}_p(t) &= \knownVar{a_p} x_p(t)
                    + \knownVar{k_p} \unknownVar{u(t)},
            \tag{\ref{eq:G-linear}}\\
            &&& \txt{IC } x_p(0) \in \R
\end{alignat*}~

Pick $u(t)$ such that the dynamical behaviour
of the closed loop is equal to that of the model.
This is done by comparing \eqn{eq:G-linear} to \eqn{eq:G-des}.
\begin{align*}
u^*(t)
    &= \frac{1}{k_p} \left( -a_p x_p(t) + a_m x_p(t) + k_m r(t) \right)\\
    &= \underbrace{\frac{a_m - a_p}{k_p}}_{a^*} x_p(t)
        + \underbrace{\frac{k_m}{k_p}}_{k^*} r(t)\\
%    &= a^* x_p(t) + k^* r(t) \\
    &= \matr{a^* & k^*} \matr{x_p(t) \\ r(t)}\\
u^*(t)
    &= \bm{\theta}^{*T} \bm{\phi}(t)
%    \numberthis \label{eq:u-perfect-linear}
\end{align*}~

Using this input%
    \footnote{%
        The starred variables with $*$ superscripts represent
        the ideal values of the control parameters.
    }%
, now the dynamics of the plant $G$ matches the dynamics
of the model $M$, as in equation \eqn{eq:G-des}.\\

However, even though now the dynamics are the same,
the initial conditions are not necessarily the same.
Would this input also work for $x_m(0) \neq x_p(0)$?
I.e., does this guarantee that
$x_p(t) \rightarrow x_m(t)$ for $t \rightarrow \infty$?

\paragraph{Dependence on the initial conditions}
To check this, we examine the error dynamics
and see if the error asymptotically goes to zero.
\begin{align*}
e(t) &= x_p(t) - x_m(t)\\
\dot{e}(t) &= \dot{x}_p(t) - \dot{x}_m(t)\\
\end{align*}

Using eqns. \eqn{eq:M} and \eqn{eq:G-des}:
\begin{align}
\dot{e}(t)    &= a_m e(t)
\label{eq:error-dynamics}
\end{align}~

\begin{conclusion}{darkgreen}
If $a_m <0$, the error dynamics are stable.
That is, $e(t) \rightarrow 0$ for any ICs.
\end{conclusion}

\paragraph{Conclusion}
\begin{itemize}
\item Model reference control (MRC) works with the error dynamics of the
    reference model
\item \redtext{We need to know all plant parameters very well} \\
    $\Rightarrow$ Problem: \redtext{uncertainty in parameters}
\end{itemize}


\subsection{Model reference adaptive control (MRAC)}
The \redtext{plant parameters} are unknown.
We assume $k_p > 0$.
\begin{alignat*}{3}
    G&:~    & \dot{x}_p(t) &= \unknownVar{a_p} x_p(t)
                    + \unknownVar{k_p} \unknownVar{u(t)},
            \tag{\ref{eq:G-linear}}\\
            &&& \txt{IC } x_p(0) \in \R
\end{alignat*}

\paragraph{Control law}
We search for (learn) the value of $\theta$ and $k$,
which are therefore functions of time.
\begin{alignat*}{3}
u(t)    %&= a(t) x_p(t) + k(t) r(t)\\
        &= \matr{a(t) & k(t)} \matr{x_p(t) \\ r(t)}\\
        &= \bm{\theta}^T(t)  \bm{\phi}(t)
\numberthis \label{eq:u-linear}
\end{alignat*}

\paragraph{Adaptive law}
Adapt the control parameters in the following fashion.
\begin{alignat*}{3}
\matr{\dot{a}(t) \\ \dot{k}(t)}
    &= - \sgn{k_p} e(t)
            \matr{\gamma_1 & 0\\
                  0 & \gamma_2} \matr{x_p(t) \\ r(t)}\\
\Rightarrow~ \dot{\bm{\theta}} &= - \sgn{k_p} e(t) \bm{\Gamma} \bm{\phi}(t)
    \numberthis \label{eq:adaptive-law}
\end{alignat*}~

The equations in \eqn{eq:adaptive-law} are nonlinear ODEs.

\paragraph{Questions}
\begin{itemize}
\item Is the closed loop stable?
\item Does, with this, $e(t) \rightarrow 0$?
\item Are the parameters $\bm{\theta}(t)$ finite? 
\item Are the parameters $\bm{\theta}(t)$ constant for $t \rightarrow \infty$?
\item Do the parameters $\bm{\theta}(t)$
        approach their `ideal' values $\bm{\theta}^*$
        for $t \rightarrow \infty$?
\end{itemize}


\section{Nonlinear SISO plant}
The linear SISO plant in the previous chapter is linearised to
a nonlinear SISO plant.\\

\begin{problem}{Control problem}
\paragraph{Given} plant $G$ and reference model $M$.
\begin{alignat*}{3}
G&:~ &
    \dot{x}_p(t) &= \unknownVar{a_p} x_p(t) + \unknownVar{k_p} u(t) \\
          &&& \quad \quad + \unknownVar{\alpha_p} f(\knownVar{z})
    \numberthis \label{eq:G-nonlinear}\\
M&:~ &
    \dot{x}_m(t) &= \knownVar{a_m} x_m(t) + \knownVar{k_m} r(t)
    \tag{\ref{eq:M}}
\end{alignat*}

\begin{itemize}
\item \redtext{$a_p$, $k_p$, $\alpha_p$} are unknown but constant
\item $f(z)$ is a nonlinear (external) function
\item $\sgn{k_p}$, $f(.)$ are known ($\knownVar{z}$ is a known signal)
\item $\alpha_m f(z)$ is not necessary
\end{itemize}

\paragraph{Goal}
$x_p(t) \rightarrow x_m(t)$ for $t \rightarrow \infty$.
\end{problem}

\subsection{Control structures}
\paragraph{Ideal control structure} based on MRC
\begin{alignat*}{3}
u^*(t)
    &= \frac{1}{k_p} \left( -a_p x_p(t) + a_m x_p(t) \right.\\
            & \left. \qquad \qquad + k_m r(t) - \alpha_p f(z) \right)\\
    &= \underbrace{\frac{\left( a_m - a_p \right)}{k_p}}_{a^*} x_p
        + \underbrace{\frac{k_m}{k_p}}_{k^*} r(t)
        + \underbrace{\frac{-\alpha_p}{k_p}}_{\alpha^*} f(z)\\
    &= \matr{a^* & k^* & \alpha^*} \matr{x_p(t) \\ r(t) \\ f(z)}\\
u^*(t)
    &= \bm{\theta}^{*T} \bm{\phi}(t)
    \numberthis \label{eq:u-ideal-nonlinear}
\end{alignat*}

\paragraph{Control law} using MRAC
\begin{alignat*}{3}
u(t) %&= a(t) x_p(t) + k(t) r(t) + \alpha(t) f(z) \\
     &= \unknownVar{\matr{a(t) & k(t) & \alpha(t)}} \matr{x_p(t) \\ r(t) \\ f(z)}\\
u(t) &= \bm{\unknownVar{\theta}}^T(t) \bm{\phi}(t)
    \numberthis \label{eq:u-nonlinear}
\end{alignat*}
\redtext{$a(t),~ k(t),~ \alpha(t)$} unknown.

\subsection{Error dynamics}
In adaptive control, the current estimated parameters are varying.
We therefore have
the following error in parameters
as deviations from the unknown but ideal and constant
real parameters:
\begin{alignat*}{3}
\left.
\begin{array}{rl}
    \tilde{a}(t) &= a(t) - a^*\\
    \tilde{k}(t) &= k(t) - k^*\\
    \tilde{\alpha}(t) &= \alpha(t) - \alpha^*
\end{array}
\right\rbrace \tilde{\bm{\theta}}(t) = \bm{\theta}(t) - \bm{\theta}^*
\end{alignat*}%
%
\begin{alignat*}{3}
\dot{e}(t) &= \dot{x}_p(t) - \dot{x}_m(t)\\
        &= a_p x_p(t)\\
        & \qquad + k_p \left( a(t) x_p(t) + k(t) r(t) + \alpha(t) f(z) \right)\\
        & \qquad + \alpha_p f(z)\\
        & \qquad - \left( a_m x_m(t) + k_m r(t) \right)\\
    &= a_p x_p(t) - a_m x_m(t)\\
        & \qquad + k_p a(t) x_p(t)\\
        & \qquad + k_p \underbrace{ \left(  k(t) - \frac{k_m}{k_p} \right) }_{ \tilde{k}(t)} r(t)\\
        & \qquad + k_p \underbrace{\left( \alpha(t) - \frac{\alpha_p}{k_p} \right)}_{\tilde{\alpha}(t)} f(z)\\
    &= \underbrace{\left( a_m - k_p a^* \right)}_{a_p} x_p(t)
        - a_m x_m(t) \\
        & \qquad + k_p a(t) x_p(t) \\
        & \qquad + k_p \tilde{k}(t) r(t)
        + k_p \tilde{\alpha}(t) f(z)\\
    &= \knownVar{a_m} e(t)
        + k_p \tilde{a}(t) x_p(t)\\
        & \qquad + k_p \tilde{k}(t) r(t)
        + k_p \tilde{\alpha}(t) f(z)
\end{alignat*}
\begin{alignat*}{3}
\dot{e}(t) &= \knownVar{a_m} \measuredVar{e(t)}
        + \unknownVar{k_p \matr{\tilde{a}(t) & \tilde{k(t)} & \tilde{\alpha}(t)}}
        \measuredVar{\matr{x_p(t) \\ r(t) \\ f(z)}} \\
\dot{e}(t)
    &= \knownVar{a_m} \measuredVar{e(t)}
        + \frac{1}{ \unknownVar{k^*}} \knownVar{k_m} \unknownVar{\bm{\theta}^T(t)} \measuredVar{\bm{\phi}(t)} 
    \numberthis \label{eq:error-dynamics-nonlinear}
\end{alignat*}~

The error dynamics can be rewritten using an operator
$ \knownVar{M(s)} = \dfrac{ \knownVar{k_m}}{s - \knownVar{a_m}}$.
\begin{alignat*}{3}
\dot{e}(t)
    &= \knownVar{a_m} \measuredVar{e(t)}
        + \unknownVar{\frac{1}{k^*} k_m \bm{\theta}^T(t)} \measuredVar{\bm{\phi}(t)} \\
\left( s - \knownVar{a_m} \right) e(t)
    &= \frac{1}{ \unknownVar{k^*}} \knownVar{k_m} \unknownVar{\bm{\theta}^T(t)} \measuredVar{\bm{\phi}(t)} \\
e(t)    &= \frac{1}{ \unknownVar{k^*}} \knownVar{M(s)} \unknownVar{\bm{\theta}^T(t)} \measuredVar{\bm{\phi}(t)}
    \tag{\ref{eq:error-dynamics-nonlinear}}
\end{alignat*}

All unknown parameters appear linearly (affine,
`linear in the parameters').
The error dynamics \eqn{eq:error-dynamics-nonlinear} is a nonlinear differential equation.
When is it stable? $ \rightarrow$ Lyapunov.


\section{Lyapunov-like functions}
\paragraph{New interpretation of Lyapunov}
Nothing to do with energy.
$\bm{V}$ affects the scaling of the distance of $\bm{x}$ 
from the origin in the phase portrait.
\begin{align*}
\norm{\bm{x}}^2_{\bm{V}} = \bm{x}^T \bm{V} \bm{x}, \qquad \bm{V} \succ 0
\end{align*}~

\begin{conclusion}{darkgreen}
\textbf{All Lyapunov says is}: how far is $\bm{x}$ 
from the origin? We want to find some type of measure
for that.
\end{conclusion}

\paragraph{Lyapunov function} (Lyapunov-like)\\
We want the output error $e(t)$ as well as the parameter
error $\bm{ \tilde{\theta}}(t)$ to go to zero.
$\bm{\Gamma} \succ 0$ symmetrical, positive definite.
\begin{alignat*}{3}
V(e, \tilde{\bm{\theta}})
    &= \frac{1}{2}e^2 + \frac{1}{2}|k_p| \left( \tilde{\bm{\theta}}^T \bm{\Gamma}^{-1} \tilde{\bm{\theta}} \right)
    \numberthis \label{eq:V}\\
\dot{V}
    &= e \dot{e} + \frac{1}{2}|k_p|
    \undertxt{\left( 2 \tilde{\bm{\theta}}^T \bm{\Gamma}^{-1} \dot{\tilde{\bm{\theta}}} \right)}{\footnotemark}
\end{alignat*}
\footnotetext{Possible due to $\bm{\Gamma}$ symmetrical.}

Substitute $\dot{e}$ using equation \eqn{eq:error-dynamics-nonlinear}.
\begin{alignat*}{3}
\dot{V}  &= a_me^2 + e k_p \tilde{\bm{\theta}}^T \bm{\phi}
        + \frac{1}{2} |k_p| \left( 2 \tilde{\bm{\theta}^T \bm{\Gamma}^{-1} \dot{\tilde{\bm{\theta}}}} \right)\\
    &= a_me^2 + |k_p| \tilde{\bm{\theta}}^T
        \underbrace{\left(
        \sgn{k_p} e \phi + \bm{\Gamma}^{-1} \dot{\tilde{\theta}}\right)}_{
        \overset{!}{=} 0}
\end{alignat*}

The second term is set to 0, because we want $V \preceq 0$ and
we don't know all the signs of the terms.
This will define the \textbf{adaptive law}.
\begin{alignat*}{3}
\dot{\tilde{\theta}}(t) &= - \sgn{k_p} \bm{\Gamma} \bm{\phi}(t) e(t)\\
\dot{\theta}(t) &= - \sgn{k_p} \bm{\Gamma} \bm{\phi}(t) e(t)\\
\end{alignat*}

With the adaptive law, we obtain for $\dot{V}$:
\begin{alignat}{3}
\dot{V} &= a_m e^2 \preceq 0 \label{eq:Vdot}
\end{alignat}

\paragraph{Remark}
$e(t)$ does not have to be 0 -- why?
% \begin{itemize}
% \item We do not need $\dot{V}$ to approach zero.
% \item $\dot{V}=0$ does not imply that $V$ has a limit as $t \rightarrow \infty$.
%     (although this is known, see footnote below%
%     \footnote{A function $V$ that is bounded from below $V \succeq 0$
%         and \textbf{non-increasing} $\dot{V} \preceq 0$
%         has a limit as $t \rightarrow \infty$}).
% \item In other words, $\dot{V}=0$ does not imply that the errors go to zero,
%     and vice versa.
% \end{itemize}~  

\begin{conclusion}{darkred}
If the derivative of a function $\rightarrow 0$,
that \redtext{does not} imply that the function
has a limit, and vice versa:
if a function has a limit, that doesn't mean its
derivative $\rightarrow 0$.
\begin{alignat*}{3}
\lim_{t \rightarrow \infty} \dot{f}(t) = 0
    \nLeftrightarrow \lim_{t \rightarrow \infty} f(t) = k
\end{alignat*}
\end{conclusion}~

Counterexamples:
\begin{align*}
f(t) &= \sin \left( \ln t \right)\\
\nexists \lim_{t \rightarrow \infty} f(t), \quad
\dot{f}(t) &= \frac{\cos \left( \ln t \right)}{t} \rightarrow 0\\
~\\
f(t) &= e^{-t} \sin (e^{2t})\\
 \lim_{t \rightarrow \infty} f(t) &= 0\\
\dot{f}(t) &= -e^{-t}\sin(e^{2t}) + e^t \sin(e^{2t}) \\
    & \quad \rightarrow    \txt{ explodes!}
\end{align*}

\paragraph{Are the error dynamics stable?}
\begin{itemize}
\item Measure (some of) the states
\item Apply $V = f(e, \tilde{\bm{\theta}})$
\item $V \rightarrow \infty$? Or $V \downarrow$?\\
    $\Rightarrow$ analyse time derivative $\dot{V}$
\item If we show $\dot{V} \rightarrow 0$, then $e \rightarrow 0$.
\end{itemize}

\paragraph{Extensions to Lyapunov}
There are two well-known extensions to Lyapunov to prove
asymptotic stability, even if $\dot{V} \preceq 0$.
\begin{enumerate}
\item LaSalle's invariance principle
    (\redtext{only for autonomous systems})
\item Barbalat's lemma
    (\greentext{OK for non-autonomous systems})
\end{enumerate}~

Our system's error dynamics are non-autonomous,
$\dot{e} = f(t, \dots)$, due to
following another system (Figure \ref{fig:non-autonomous-system}).
\begin{figure}[H]
\centering
\inkscape[\normalsize]{e-non-auto}{0.25}
\caption{Non-autonomous dynamics}
\label{fig:non-autonomous-system}
\end{figure}



\section{Barbalat's lemma}
\subsection{Uniformly continuous functions}
\label{ch:uniform-continuous}
\begin{definition}{Uniformly continuous function}
A function $f(t): \R \rightarrow \R$ 
is uniformly continous, if\\
$\forall \varepsilon>0:
    \quad \exists~ \delta = \delta(\varepsilon) >0,$
\begin{alignat*}{3}
\forall |t_2 - t_1| &\leq \delta\\
\Rightarrow |  f(t_2) - f(t_1) | &\leq \varepsilon
\end{alignat*}
%\label{ch:uniform-continuous}
\end{definition}

\begin{conclusion}{darkgreen}
\textbf{Sufficient condition for uniformly continuous functions}:
If the derivative $\dot{f}(t)$ exists (i.e. bounded),
$\Rightarrow f(t)$ is uniformly constant.
\end{conclusion}~

\subsection{Barbalat's lemma}
\label{ch:barbalat}
\begin{lemma}{Barbalat Variant A}
If $f(t): \R \rightarrow \R$
\begin{enumerate}
\item is a differentiable function
\item has a finite limit as $t \rightarrow \infty$ 
\item $\dot{f}(t)$ is uniformly continuous
\end{enumerate}
$\Rightarrow \lim_{t \rightarrow \infty} \dot{f}(t) = 0$\\
\end{lemma}

\begin{lemma}{Barbalat Variant B}
If
\begin{enumerate}
\item $f(t): \R \rightarrow \R$ is uniformly continous $\forall t$
\item $\exists~ \lim_{t \rightarrow \infty} \int_0^t f(\tau) d\tau$
\end{enumerate}
$\Rightarrow \lim_{t \rightarrow \infty} f(t) = 0$\\
\end{lemma}

\begin{lemma}{Barbalat Variant C}
If $f, \dot{f} \in \mathcal{L}_\infty$ 
and $f \in \mathcal{L}_2$,
then  $|f(t)| \rightarrow 0$ as $t \rightarrow \infty$.
\end{lemma}


\begin{definition}{Signal norm}
\paragraph{Idea} quantify magnitude of a signal $x(t)$
-- ``How big is a signal?'' -- also called the $\mathcal{L}_p$ space.\\

\paragraph{Given}
\begin{alignat*}{3}
x(t): \R^+ \rightarrow \R^n, \quad  \quad \R^+ = [0, \infty)
\end{alignat*}

\paragraph{p-Norm}
    \begin{alignat*}{3}
    \|x_p\| &= \left( \int_0^\infty |x(t)|^p dt \right)^{1/p}
        & \qquad p \in (0, \infty)
    \end{alignat*}
\end{definition}~

\begin{itemize}
\item If $\bm{x}(t)$ vector, $|.|$ is the vector 2-norm,
    `\textbf{distance}'.
\item $\|x\|_\infty = \sup_{t \in \R^+} |x(t)| \corresponds$ 
    highest value of $\bm{x}$\\
    ``When power $\infty$, only the greatest value survives''
\end{itemize}~

\begin{definition}{Functional space}
\begin{alignat*}{3}
\mathcal{L}_p = \{x(t) \in \R^n : \undertxt{\| x \|_p < \infty}{exists} \}
\end{alignat*}
\end{definition}~

\begin{example}{Functional space}
\begin{itemize}
\item $x(t) \in \mathcal{L}_p$: $x$ is bounded.\\
``$x$'s highest value exists and is not infinity.''
\item Show that $e \in \mathcal{L}_\infty$ \\
$e$ is in $V$ and $V$ is bounded, $\therefore e \in \mathcal{L}_\infty$.
\item Show that $e \in \mathcal{L}_2$ 
\begin{alignat*}{3}
\int_0^\infty \dot{V} dt &= V(\infty) - V(0), \qquad \txt{is bounded}.\\
a_m \int_0^\infty e^2 dt &\in \mathcal{L}_\infty\\
\Rightarrow e &\in \mathcal{L}_2
\end{alignat*}
\end{itemize}
\end{example}



%% TODO FILL IN


% \subsection{Kalman-Yakobovich Lemma (KY)}
% ``Maier version; how to design a controller given SPR''\\
% 
% \begin{itemize}
% \item Given a scalar $\gamma \geq 0$, vectors $b$ and $c$,
%     an asymptotically stable matrix $A$ %
%     \footnote{positive eigenvalues}
%     and a positive definite matrix $L$,
% \item If $H(s) \corresponds \frac{1}{2} \gamma + c^T (sI-A)^{-1} b$ \\
%     $\Rightarrow H(s)$ is SPR
% \item Then, $\exists$ a scalar $\varepsilon >0$, a vector $q$ 
%     and a symmetric positive definite matrix $P$, s.t.
%     \begin{alignat*}{3}
%     A^TP + PA   &= -qq^T - \varepsilon L\\
%     Pb - c      &= \sqrt{\gamma} q
%     \end{alignat*}
% \end{itemize}
% 
% \paragraph{Using it}
% We only need $\gamma=0$ in all cases (in this course).
% Hence we can say: if $H(s)$ is SPR $\Rightarrow \exists P=P^T>0$,
% s.t.
% \begin{alignat*}{3}
% A^TP + PA   &= -Q\\
% Pb &c   \quad \txt{"boundary condition, means SPR"}
% \end{alignat*}
% where $Q=Q^t>0$
% 
% % \subsection{Lemma: ``Adaptive laws based on Lyapunov for
% % relative degree 1 plants''}
% % 
% % Consider the dynamical system below%
% % \footnote{think of the error dynamics, not of the plant!}.
% % \begin{alignat*}{3}
% % \dot{x}(t) &= A x(t) + b \theta^T(t) \phi(t) \quad & x &\in \R^n\\
% % y(t) &= c^Tx(t) \quad           & y, z_1 &\in \R^1\\
% % z_1(t) &= ky(t)             & \phi, \theta & \in \R^k
% % \end{alignat*}
% % ``$z_1$ allows change of symbol.\\
% % 
% % where $(A,b)$ is stabilisable and $(c^T,A)$ 
% % is detectable and $c^T(sI-A)^{-1}b \corresponds H(s)$ 
% % is SPR.\\
% %  
% % Let $\theta(t)$ be a vector of adjustable parameters.\\
% % Let $\phi(t)$ and $z_1(t)$ be time-varying functions that can
% % be measured.\\
% % 
% % Then, if $\theta(t)$ is adjusted as
% % \begin{alignat}{3}
% % \dot{\theta}(t) &= -\sign{k} z_1(t) \phi(t)
% % \end{alignat}
% % (Adaptive law for multistate systems)
% % 
% % $\Rightarrow$ the equilibrium state $(x=0, \theta=0)$ 
% % is uniformly stable in the large.\\
% % uniformly stable: not dependent on time nor on IC\\
% % in the large: IC don't matter everywhere in $\R^n$.
% % 
% % \paragraph{Proof}
% % Since $H(s)$ is SPR, it follows from KY lemma that $\exists P = P^T >0$ 
% % such that
% % \begin{alignat}{3}
% % A^TP + PA = -Q, \quad \txt{with } Q^T = Q>0\\
% % Pb &= c
% % \end{alignat}
% % 
% % Let $V$ be a positive definite function
% % \begin{alignat}{3}
% % V &= x^TPx + \frac{1}{|k|} \theta^T \theta
% % \end{alignat}
% % 
% % Its time derivative along the system is
% % \begin{alignat}{3}
% % \dot{V} &= x^T \left( PA + A^TP \right)x + 2 x^TPb \theta^T\phi - 2\theta^Ty\phi\\
% %         &= -x^TQx \leq 0
% % \end{alignat}
% % 
% % Therefore, the origin of (1), (2) is stable.
% % %% TODO correct references.
% % 
% % \paragraph{Discussion}
% % We now have a simple tool for finding adaptive laws with
% % error dynamics
% % \begin{alignat*}{3}
% % e &= \frac{1}{k^*} M(s) [ \tilde{\theta}^T \phi]
% % \end{alignat*}
% % (where $M(s)$ SPR(s)tabilisable and detectable)
% % and the adaptive laws are
% % \begin{alignat}{3}
% % \dot{\theta} &= -\sign{\varepsilon} e \Gamma \phi
% % \end{alignat}
% % 
% % If $M(s)$ has relative degree 1, it is obvious that
% % $e(t) \rightarrow 0$ for $t \rightarrow \infty$.
% % (Why?)
% % 
% % \subsection{Performance considerations}
% % Engineering criteria:\\
% % \begin{tabu} to \columnwidth{|c|c|}
% % \toprule
% % Performance & Noise\\
% % \midrule
% % Disturbance & Robustness
% % \bottomrule
% % \end{tabu}
% % 
% % \begin{itemize}
% % \item Increasing $\gamma$, we are unhappy with the oscillations of our 
% %     parameters $\theta(t)$ and therefore with the
% %     oscillations of $u(t)$.
% % \item We have no clue what the adaptive closed loop will do between
% %     $t=0$ and $t=\infty$ other than boundedness.
% % \end{itemize}
% % 
% % \paragraph{Now}
% % Deal with transient response.
% % 
% % \paragraph{Idea}
% % Adaptation changes with signals.
% % \begin{alignat*}{3}
% % \left( \dot{\theta}(t) = - \sign{\varepsilon} e \gamma \phi \right),
% % \quad e, \gamma \txt{ changeable}
% % \end{alignat*}
% % 
% % We can alter the transient with $\gamma$ (leads to oscillations),
% % or we can change $e(t)$.
% % 
% % \paragraph{So far} Open loop reference model (ORM)
% % \begin{alignat}{3}
% % \dot{x}_m(t)^o &= a_m x_m(t)^o + k_m r(t)\\
% % \end{alignat}
% % 
% % %% TODO control block
% % 
% % \paragraph{Now} Closed loop reference model (CRM)
% % \begin{alignat}{3}
% % \dot{x}_m(t)^c &= a_m x_m(t)^c + k_m r(t) - l e^c(t)
% % \end{alignat}
% % 
% % ``CRM is observer-like; $M$ helps $G$ by moving towards
% % $G$ and retreating to original position.''
% % 
% % ORM $=$ CRM if $l=0$.
