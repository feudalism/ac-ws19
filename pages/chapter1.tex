\section{Linear SISO plant}
\section{Linear SISO plant}
\begin{problem}{Control problem}
\paragraph{Given} plant $G$ and reference model $M$.
\begin{alignat}{3}
    G&:~    & \dot{x}_p(t) &= a_p x_p(t) + k_p \unknownVar{u(t)},
            \label{eq:G-linear}\\
            &&& \txt{IC } x_p(0) \in \R \nonumber
\end{alignat}
\begin{variables}
    a_p     & pole of plant\\
    k_p     & input gain of plant\\
\end{variables}

\begin{alignat}{3}
    M&:~    & \dot{x}_m(t) &= \knownVar{a_m} x_m(t)
                        + \knownVar{k_m} \knownVar{r(t)},
            \label{eq:M}\\
            &&& \txt{IC } x_m(0) \in \R \nonumber
\end{alignat}
\begin{variables}
    \knownVar{a_m}     & pole of reference model\\
    \knownVar{k_m}     & input gain of reference model\\
    \knownVar{r(t)}    & reference signal
\end{variables}%
%
The \greentext{reference model parameters} are set by the user
and are therefore known.

\paragraph{Task} find a control $u(t)$ such that
$x_p(t) \rightarrow x_m(t)$ for $t\rightarrow \infty$.
\begin{alignat}{3}
    \tund{G}{des}:~
    && \dot{x}_p(t)    &= \knownVar{a_m} x_p(t)
                    + \knownVar{k_m r(t)}
                \label{eq:G-des}\\
                &&&\txt{IC } x_p(0) \in \R \nonumber
\end{alignat}
\end{problem}
~

Solutions for $u(t)$ using
\begin{itemize}
\item Model reference control (MRC)
\item Model reference adaptive control (MRAC)
\end{itemize}

\subsection{Model reference control (MRC)}
The \greentext{plant parameters} are assumed to be known.
\begin{alignat*}{3}
    G&:~    & \dot{x}_p(t) &= \knownVar{a_p} x_p(t)
                    + \knownVar{k_p} \unknownVar{u(t)},
            \tag{\ref{eq:G-linear}}\\
            &&& \txt{IC } x_p(0) \in \R
\end{alignat*}~

Pick $u(t)$ such that the dynamical behaviour
of the closed loop is equal to that of the model.
This is done by comparing \eqn{eq:G-linear} to \eqn{eq:G-des}.
\begin{align*}
u^*(t)
    &= \frac{1}{k_p} \left( -a_p x_p(t) + a_m x_p(t) + k_m r(t) \right)\\
    &= \underbrace{\frac{a_m - a_p}{k_p}}_{a^*} x_p(t)
        + \underbrace{\frac{k_m}{k_p}}_{k^*} r(t)\\
%    &= a^* x_p(t) + k^* r(t) \\
    &= \matr{a^* & k^*} \matr{x_p(t) \\ r(t)}\\
u^*(t)
    &= \bm{\theta}^{*T} \bm{\phi}(t)
%    \numberthis \label{eq:u-perfect-linear}
\end{align*}~

Using this input%
    \footnote{%
        The starred variables with $*$ superscripts represent
        the ideal values of the control parameters.
    }%
, now the dynamics of the plant $G$ matches the dynamics
of the model $M$, as in equation \eqn{eq:G-des}.\\

However, even though now the dynamics are the same,
the initial conditions are not necessarily the same.
Would this input also work for $x_m(0) \neq x_p(0)$?
I.e., does this guarantee that
$x_p(t) \rightarrow x_m(t)$ for $t \rightarrow \infty$?

\paragraph{Dependence on the initial conditions}
To check this, we examine the error dynamics
and see if the error asymptotically goes to zero.
\begin{align*}
e(t) &= x_p(t) - x_m(t)\\
\dot{e}(t) &= \dot{x}_p(t) - \dot{x}_m(t)\\
\end{align*}

Using eqns. \eqn{eq:M} and \eqn{eq:G-des}:
\begin{align}
\dot{e}(t)    &= a_m e(t)
\label{eq:error-dynamics}
\end{align}~

\begin{conclusion}{darkgreen}
If $a_m <0$, the error dynamics are stable.
That is, $e(t) \rightarrow 0$ for any ICs.
\end{conclusion}

\paragraph{Conclusion}
\begin{itemize}
\item Model reference control (MRC) works with the error dynamics of the
    reference model
\item \redtext{We need to know all plant parameters very well} \\
    $\Rightarrow$ Problem: \redtext{uncertainty in parameters}
\end{itemize}


\subsection{Model reference adaptive control (MRAC)}
The \redtext{plant parameters} are unknown.
We assume $k_p > 0$.
\begin{alignat*}{3}
    G&:~    & \dot{x}_p(t) &= \unknownVar{a_p} x_p(t)
                    + \unknownVar{k_p} \unknownVar{u(t)},
            \tag{\ref{eq:G-linear}}\\
            &&& \txt{IC } x_p(0) \in \R
\end{alignat*}

\paragraph{Control law}
We search for (learn) the value of $\theta$ and $k$,
which are therefore functions of time.
\begin{alignat*}{3}
u(t)    %&= a(t) x_p(t) + k(t) r(t)\\
        &= \matr{a(t) & k(t)} \matr{x_p(t) \\ r(t)}\\
        &= \bm{\theta}^T(t)  \bm{\phi}(t)
\numberthis \label{eq:u-linear}
\end{alignat*}

\paragraph{Adaptive law}
Adapt the control parameters in the following fashion.
\begin{alignat*}{3}
\matr{\dot{a}(t) \\ \dot{k}(t)}
    &= - \sgn{k_p} e(t)
            \matr{\gamma_1 & 0\\
                  0 & \gamma_2} \matr{x_p(t) \\ r(t)}\\
\Rightarrow~ \dot{\bm{\theta}} &= - \sgn{k_p} e(t) \bm{\Gamma} \bm{\phi}(t)
    \numberthis \label{eq:adaptive-law}
\end{alignat*}~

The equations in \eqn{eq:adaptive-law} are nonlinear ODEs.

\paragraph{Questions}
\begin{itemize}
\item Is the closed loop stable?
\item Does, with this, $e(t) \rightarrow 0$?
\item Are the parameters $\bm{\theta}(t)$ finite? 
\item Are the parameters $\bm{\theta}(t)$ constant for $t \rightarrow \infty$?
\item Do the parameters $\bm{\theta}(t)$
        approach their `ideal' values $\bm{\theta}^*$
        for $t \rightarrow \infty$?
\end{itemize}




\section{Nonlinear SISO plant}
The linear SISO plant in the previous chapter is linearised to
a nonlinear SISO plant.\\

\begin{problem}{Control problem}
\paragraph{Given} plant $G$ and reference model $M$.
\begin{alignat*}{3}
G&:~ &
    \dot{x}_p(t) &= \unknownVar{a_p} x_p(t) + \unknownVar{k_p} u(t) \\
          &&& \quad \quad + \unknownVar{\alpha_p} f(\knownVar{z})
    \numberthis \label{eq:G-nonlinear}\\
M&:~ &
    \dot{x}_m(t) &= \knownVar{a_m} x_m(t) + \knownVar{k_m} r(t)
    \tag{\ref{eq:M}}
\end{alignat*}

\begin{itemize}
\item \redtext{$a_p$, $k_p$, $\alpha_p$} are unknown but constant
\item $f(z)$ is a nonlinear (external) function
\item $\sgn{k_p}$, $f(.)$ are known ($\knownVar{z}$ is a known signal)
\item $\alpha_m f(z)$ is not necessary
\end{itemize}

\paragraph{Goal}
$x_p(t) \rightarrow x_m(t)$ for $t \rightarrow \infty$.
\end{problem}

\subsection{Control structures}
\paragraph{Ideal control structure} based on MRC
\begin{alignat*}{3}
u^*(t)
    &= \frac{1}{k_p} \left( -a_p x_p(t) + a_m x_p(t) \right.\\
            & \left. \qquad \qquad + k_m r(t) - \alpha_p f(z) \right)\\
    &= \underbrace{\frac{\left( a_m - a_p \right)}{k_p}}_{a^*} x_p
        + \underbrace{\frac{k_m}{k_p}}_{k^*} r(t)
        + \underbrace{\frac{-\alpha_p}{k_p}}_{\alpha^*} f(z)\\
    &= \matr{a^* & k^* & \alpha^*} \matr{x_p(t) \\ r(t) \\ f(z)}\\
u^*(t)
    &= \bm{\theta}^{*T} \bm{\phi}(t)
    \numberthis \label{eq:u-ideal-nonlinear}
\end{alignat*}

\paragraph{Control law} using MRAC
\begin{alignat*}{3}
u(t) %&= a(t) x_p(t) + k(t) r(t) + \alpha(t) f(z) \\
     &= \unknownVar{\matr{a(t) & k(t) & \alpha(t)}} \matr{x_p(t) \\ r(t) \\ f(z)}\\
u(t) &= \bm{\unknownVar{\theta}}^T(t) \bm{\phi}(t)
    \numberthis \label{eq:u-nonlinear}
\end{alignat*}
\redtext{$a(t),~ k(t),~ \alpha(t)$} unknown.

\subsection{Error dynamics}
In adaptive control, the current estimated parameters are varying.
We therefore have
the following error in parameters
as deviations from the unknown but ideal and constant
real parameters:
\begin{alignat*}{3}
\left.
\begin{array}{rl}
    \tilde{a}(t) &= a(t) - a^*\\
    \tilde{k}(t) &= k(t) - k^*\\
    \tilde{\alpha}(t) &= \alpha(t) - \alpha^*
\end{array}
\right\rbrace \tilde{\bm{\theta}}(t) = \bm{\theta}(t) - \bm{\theta}^*
\end{alignat*}%
%
\begin{alignat*}{3}
\dot{e}(t) &= \dot{x}_p(t) - \dot{x}_m(t)\\
        &= a_p x_p(t)\\
        & \qquad + k_p \left( a(t) x_p(t) + k(t) r(t) + \alpha(t) f(z) \right)\\
        & \qquad + \alpha_p f(z)\\
        & \qquad - \left( a_m x_m(t) + k_m r(t) \right)\\
    &= a_p x_p(t) - a_m x_m(t)\\
        & \qquad + k_p a(t) x_p(t)\\
        & \qquad + k_p \underbrace{ \left(  k(t) - \frac{k_m}{k_p} \right) }_{ \tilde{k}(t)} r(t)\\
        & \qquad + k_p \underbrace{\left( \alpha(t) - \frac{\alpha_p}{k_p} \right)}_{\tilde{\alpha}(t)} f(z)\\
    &= \underbrace{\left( a_m - k_p a^* \right)}_{a_p} x_p(t)
        - a_m x_m(t) \\
        & \qquad + k_p a(t) x_p(t) \\
        & \qquad + k_p \tilde{k}(t) r(t)
        + k_p \tilde{\alpha}(t) f(z)\\
    &= \knownVar{a_m} e(t)
        + k_p \tilde{a}(t) x_p(t)\\
        & \qquad + k_p \tilde{k}(t) r(t)
        + k_p \tilde{\alpha}(t) f(z)
\end{alignat*}
\begin{alignat*}{3}
\dot{e}(t) &= \knownVar{a_m} \measuredVar{e(t)}
        + \unknownVar{k_p \matr{\tilde{a}(t) & \tilde{k(t)} & \tilde{\alpha}(t)}}
        \measuredVar{\matr{x_p(t) \\ r(t) \\ f(z)}} \\
\dot{e}(t)
    &= \knownVar{a_m} \measuredVar{e(t)}
        + \frac{1}{ \unknownVar{k^*}} \knownVar{k_m} \unknownVar{\bm{\theta}^T(t)} \measuredVar{\bm{\phi}(t)} 
    \numberthis \label{eq:error-dynamics-nonlinear}
\end{alignat*}~

The error dynamics can be rewritten using an operator
$ \knownVar{M(s)} = \dfrac{ \knownVar{k_m}}{s - \knownVar{a_m}}$.
\begin{alignat*}{3}
\dot{e}(t)
    &= \knownVar{a_m} \measuredVar{e(t)}
        + \unknownVar{\frac{1}{k^*} k_m \bm{\theta}^T(t)} \measuredVar{\bm{\phi}(t)} \\
\left( s - \knownVar{a_m} \right) e(t)
    &= \frac{1}{ \unknownVar{k^*}} \knownVar{k_m} \unknownVar{\bm{\theta}^T(t)} \measuredVar{\bm{\phi}(t)} \\
e(t)    &= \frac{1}{ \unknownVar{k^*}} \knownVar{M(s)} \unknownVar{\bm{\theta}^T(t)} \measuredVar{\bm{\phi}(t)}
    \tag{\ref{eq:error-dynamics-nonlinear}}
\end{alignat*}

All unknown parameters appear linearly (affine,
`linear in the parameters').
The error dynamics \eqn{eq:error-dynamics-nonlinear} is a nonlinear differential equation.
When is it stable? $ \rightarrow$ Lyapunov.


\section{Positive real functions}
\begin{definition}{Positive real function I}
A rational function $H(s): \Z \rightarrow \Z$, $s = \sigma + j\omega$ is
positive real (PR), if
\begin{enumerate}[label=(\roman*)]
\item \legendsquare{babypink} $H(s)$ is real for real $s$ 
\item \legendsquare{babyblue} $\Re\{H(s)\} \geq 0$ for  $\Re\{s\} > 0$
\end{enumerate}

\begin{figure}[H]
    \centering
    \inkscape[\footnotesize]{pr-fcn}{0.81}
    \caption{Positive real mapping $s \rightarrow H(s)$}
    \label{fig:positive-real-mapping}
\end{figure}

\end{definition}

\begin{definition}{Positive real function II}
A rational function $H(s)$ is positive real, if
    \begin{enumerate}[label=(\roman*)]
    \item $H(s)$ is analytic
    %\footnote{\textbf{analytic}: ``converges to Taylor series, somewhat smooth.''}
     in $\real{s} >0$ 
        \begin{itemize}
        \item $H(s)$ has no poles in RHP ($\real{s} > 0$)
        \item $H(s)$ is stable
        \end{itemize}
    \item $\real{H(j\omega)} \geq 0, \quad \forall\omega \in [0, \infty]$    
        \begin{itemize}
            \item Nyquist of $H(s)$ is in the RHP
            \item phase $\angle H(j\omega) \in [-90^{\circ}, +90^{\circ}]$
            \item rel. degree of $H(s)$ is 0 or 1
        \end{itemize}
    \item any pure imaginary pole $j\omega$ of $H(s)$ is
        a simple pole, and the residue
    \begin{alignat*}{3}
    \lim_{s \rightarrow j\omega} \left( s - j\omega \right) H(s)
    \end{alignat*}
        is positive semidefinite.\\

        Alternatively: $H(\infty) > 0$ or 
    \begin{alignat*}{3}
        \lim_{\omega \rightarrow \infty} \omega^2 \real{H(j\omega)} \geq 0
    \end{alignat*}
    \end{enumerate}
\end{definition}~

\begin{definition}{Strictly positive real functions}
$H(s)$ is strictly positive real (SPR) if $H(s - \varepsilon)$ 
is PR for some $\varepsilon > 0$.
\end{definition}~

\paragraph{Note} relative degree of a system corresponds to
its response delay.

\begin{lemma}{SPR lemma}
$H(s)$ is SPR if
\begin{enumerate}[label=(\roman*)]
    \item $H(s)$ is Hurwitz\\
        all poles on LHP, none are purely imaginary
    \item $\real{H(j\omega)} > 0, \quad \forall \omega \in \R$
        \begin{itemize}
            \item Nyquist of $H(s)$ is in the RHP and \redtext{not on the imaginary axis}.
            \item phase $\angle H(j\omega) \in (-90^{\circ}, +90^{\circ})$
            \item rel. degree of $H(s)$ $\in \set{0,1}$
        \end{itemize}
        phase $\in (-90 ^{\circ}, 90 ^{\circ})$\\
        rel. degree $\in \set{0, 1}$
    \item $H( \infty) > 0$ (positive gain for proper $H$) or 
        \begin{alignat*}{3}
        \lim_{\omega \rightarrow \infty} \omega^2 \real{H(j\omega)} > 0
        \end{alignat*}
        positive gain for relative degree 1
\end{enumerate}
\end{lemma}

\paragraph{Discussion}
If $H(s)$ is SPR, then so is the inverse $H^{-1}(s)$
(stable poles, stable zeroes).\\

\begin{conclusion}{darkgreen}
    SPR  $\Rightarrow H(s)$ is stable, minimal-phase.\\
    I.e., only stable zeroes, because zeroes are in LHP.
    \begin{itemize}
        \item phase $\angle H(j\omega) \in (-90^{\circ}, +90^{\circ})$
        \item rel. degree of $H(s) \leq 1 (-1, 0, 1)$
        \item positive gain $\forall ~ \omega$
    \end{itemize}
\end{conclusion}~

\begin{example}{PR}
$G(s) = \frac{1}{s}$ has a single pole $s = 0$,
with a residue of 1.
\begin{alignat*}{3}
\real{G(j\omega)} = \real{ \frac{1}{j\omega}} = 0 \qquad \forall \omega \neq 0
\end{alignat*}

Hence, $G(s)$ is PR but not SPR,
as $ \frac{1}{s - \varepsilon}$ has a pole
in $\real{s} \geq 0$ for any
$\varepsilon > 0$. 
\end{example}

\begin{example}{PR}
$G(s) = \dfrac{1}{s + a}, a>0$ is Hurwitz.
\begin{alignat*}{3}
\real{G(j\omega)} &= \dfrac{a}{\omega^2 + a^2} > 0 \\
    & \qquad
    \forall \omega \in [0, \infty]\\
\lim_{\omega \rightarrow \infty} \omega^2 \real{G(j\omega)}
    &= \lim_{\omega \rightarrow \infty} \dfrac{\omega^2a}{\omega^2 + a^2}\\
    &= a, \qquad a > 0
\end{alignat*}
\end{example}

\begin{example}{PR}
\begin{alignat*}{3}
G(s) &= \dfrac{1}{s^2 + s + 1}\\
\real{G(j\omega)} &= \dfrac{1 - \omega^2}{(1 - \omega^2)^2 + \omega^2}\\
    &\ngtr 0 \qquad \forall \omega\\
    &\Rightarrow G \txt{ is not PR.}
\end{alignat*}
\end{example}




\section{Kalman-Yakobovich lemma (KY)}
``Maier version; how to design a controller given SPR''\\

\begin{lemma}{Kalman-Yakobovich Lemma}
    \paragraph{Given}
    \begin{itemize}
    \item a scalar $\gamma \geq 0$
    \item vectors $\bm{b}$ and $\bm{c}$,
    \item an asymptotically stable matrix $\bm{A}$%
        \footnote{positive eigenvalues}, and
    \item a positive definite matrix $\bm{L} \succ 0$,
    \end{itemize}
    \begin{alignat*}{3}
    \txt{If } H(s) &\corresponds \frac{1}{2} \gamma + \bm{c}^T (s\bm{I}-\bm{A})^{-1} \bm{b}\\
        \Rightarrow H(s) &\txt{ is SPR}
    \end{alignat*}~

    Then, there exist
    \begin{itemize}
    \item a scalar $\varepsilon >0$
    \item a vector $\bm{q}$, and
    \item a symmetric positive definite matrix $\bm{P}$,
    \end{itemize}
    s.t.
    \begin{alignat}{3}
    \bm{A^TP + PA}   &= -\bm{qq}^T - \varepsilon \bm{L}
        \label{eq:ky-lemma-atp}\\
    \bm{Pb} - \bm{c}      &= \sqrt{\gamma} \bm{q}
        \label{eq:ky-lemma-bc}
    \end{alignat}
\end{lemma}

\paragraph{Using it}
We only need $\gamma=0$ in all cases (in this course).
Hence we can say: if $H(s)$ is SPR $\Rightarrow \exists ~ \bm{P}=\bm{P}^T>0$,\\
s.t.
\begin{alignat*}{3}
\bm{A^TP + PA}   &= -\bm{Q}\\
\bm{Pb} &= \bm{c} \qquad \txt{(s. note)}\footnotemark  \\
\end{alignat*}%
\footnotetext{"boundary cond., means SPR"}%
where $\bm{Q}=\bm{Q}^T>0$

\begin{lemma}{Adaptive laws based on Lyapunov}
(For rel. degree 1 plants)\\
Consider the dynamical system below%
\footnote{This refers to the error dynamics, \redtext{not the plant}!}%
\footnote{$z_1$ allows change of symbol with respect to the output $y(t)$}%
\footnote{$x \in \R^n,
        \qquad y, z \in \R^1,
        \qquad \bm{\phi}, \bm{\theta} \in \R^k$}.
\begin{alignat*}{3}
\dot{\bm{x}}(t) &= \bm{A} \bm{x}(t) + \bm{b} \bm{\theta}^T(t) \bm{\phi}(t)
    \numberthis \label{eq:G-mimo}\\
y(t) &= \bm{c}^T \bm{x}(t)     \\
z_1(t) &= k y(t) 
\end{alignat*}

where
\begin{itemize}
\item $(\bm{A}, \bm{b})$ is stabilisable
\item $(\bm{c}^T, \bm{A})$ is detectable
\item $\bm{c}^T \left( s \bm{I} - \bm{A} \right)^{-1} \bm{b} \corresponds H(s)$
    is SPR
\end{itemize}~

Let $\bm{\theta}(t)$ be a vector of adjustable parameters.\\
Let $\bm{\phi}(t)$ and $z_1(t)$ be time-varying functions that can be measured.\\

Then, if $\bm{\theta}(t)$ is adjusted as
\begin{alignat}{3}
\bm{\dot{\theta}}(t) &= - \sign{k} z_1(t) \bm{\phi}(t)
\label{eq:adaptive-law-multistate}
\end{alignat}

$\Rightarrow$ the \greentext{equilibrium state} $\left( \bm{x}=0, \bm{\theta}=0 \right)$ 
is \greentext{uniformly stable%
\footnote{uniformly stable: not dependent on time nor on IC}
 in the large%
\footnote{in the large: IC don't matter anywhere in $\R^n$}}.
\end{lemma}

\paragraph{Proof}
Since $H(s)$ is SPR, it follows from
the KY-lemma that $\exists ~ \bm{P} = \bm{P}^T > 0$,
such that
\begin{alignat*}{3}
\bm{A^TP + PA}   &= -\bm{Q}, \quad \bm{Q} = \bm{Q}^T > 0
    \tag{\ref{eq:ky-lemma-atp}}\\
\bm{Pb} &= \bm{c} 
    \tag{\ref{eq:ky-lemma-bc}}
\end{alignat*}%

Let $V$ be a positive definite function
\begin{alignat*}{3}
V &= \bm{x}^T \bm{P} \bm{x} + \dfrac{1}{|k|} \bm{\theta}^T \bm{\theta}\\
\dot{V} &= \bm{x}^T \left(  \bm{PA + A^TP} \right) \bm{x}\\
        &\quad + 2 \bm{x^T Pb \theta^T \phi} - 2 \bm{\theta^T} y \bm{\phi}\\
    &= - \bm{x^TQx} \leq 0
\end{alignat*}

Therefore, the origin of the system \eqn{eq:G-mimo}
together with the adaptive law \eqn{eq:adaptive-law-multistate}
is stable.

\paragraph{Discussion}
We now have a simple tool for finding adaptive laws
with error dynamics below \eqn{eq:error-dynamics-nonlinear}, where $M(s)$ is SPR,
stabilisable and detectable,
and the adaptive laws are defined as in \eqn{eq:adaptive-law-eps}
\begin{alignat*}{3}
e(t) &= \dfrac{1}{ \unknownVar{k^*}} \knownVar{M(s)} \matr{ \unknownVar{ \tilde{\bm{\theta}}^T \bm{\phi}}}
    \tag{\ref{eq:error-dynamics-nonlinear}}\\
\dot{\bm{\theta}}(t) &= - \sign{\varepsilon} e \bm{\Gamma} \bm{\phi}
    \numberthis \label{eq:adaptive-law-eps}
\end{alignat*}

If $M(s)$ has a relative degree of 1, it is obvious that $e(t) \rightarrow 0$
for $t \rightarrow \infty$%
\footnote{Why?}.


\section{Performance considerations}
Performance criteria:
\begin{center}
\begin{tabu} to \columnwidth {cc}
Performance & Noise\\
Disturbances    & Robustness
\end{tabu}
\end{center}~

\begin{itemize}
\item Increasing $\gamma$, we are unhappy with the
    \redtext{oscillations of our parameters $\bm{\theta}$}
    and therefore with the oscillations of $u(t)$.
\item We have no clue what the adaptive closed loop
    will do between $t=0$ and $t=\infty$ other than boundedness
\end{itemize}

\subsection{Adaptation with a closed loop reference model}
\paragraph{Now} Deal with transient response
\paragraph{Idea} Adaptation changes with signals
\begin{alignat*}{3}
    \dot{\theta} &= - \sgn{\varepsilon} \measuredVar{e \gamma} \phi
\end{alignat*}
where the value of $e$ and $\gamma$ are changeable.\\
$\Rightarrow$ we can alter the transient with $\gamma$ (leads to oscillations),
\greentext{or we can change $e(t)$}.

\paragraph{So far} Open loop reference model (ORM)
\begin{alignat*}{3}
\dot{x}_m^o(t) &= a_m x_m^o(t) + k_m r(t)
    \tag{\ref{eq:M}}
\end{alignat*}

\paragraph{Now} Closed loop reference model (CRM)
\begin{alignat}{3}
\dot{x}_m^c(t) &= a_m x_m^c(t) + k_m r(t) - l e^c(t)
    \label{eq:M-crm}
\end{alignat}~

ORM $=$ CRM if $l=0$.\\

``CRM is observer-like; $M$ helps $G$ by moving towards
$G$ and retreating to original position.''\\
\redtext{Through the movement, the reference model now
has a different behaviour ($M \rightarrow M'$!)
and the plant $P$ is trying to follow $M'$.}\\

\begin{variables}
\gamma  & learning effect\\
        & decreasing $\gamma$ helps $P$ follow $M'$,\\
        & but the learning becomes slower\\
l       & movement to $P$ \\
        & increasing $l$ helps $P$ follow $M'$
\end{variables}


\subsection{Stability proof}
\begin{alignat*}{3}
\dot{x}_m^c(t) &= a_m x_m^c(t) + k_m r(t) - le^c(t)
    \tag{\ref{eq:M-crm}}\\
\dot{x}_p(t) &= a_p x_p(t) + k_p u(t) 
    \tag{\ref{eq:G-linear}}
\end{alignat*}

Input is
\begin{alignat*}{3}
u &= \matr{a(t) & k(t)} \matr{x_p(t) \\ r(t)}% =  a(t) x_p(t) + k(t) r(t)\\
    = \bm{\theta}^T(t) \bm{\phi}(t) 
\end{alignat*}

\begin{alignat*}{3}
\dot{x}_p(t) &= a_m x_p(t) + k_m r(t) + k_p \bm{\theta}^T(t) \bm{\phi}(t)
\end{alignat*}

Tracking error
\begin{alignat*}{3}
\dot{e}^c(t) &= \dot{x}_p(t) - \dot{x}_m^c(t)\\
    &= \left( a_m + l \right) e^c(t) + k_p \bm{\theta}^T(t) \bm{\phi}(t)
\end{alignat*}

Lyapunov-like function
\begin{alignat*}{3}
V(e^c, \tilde{\bm{\theta}})
    &= \frac{1}{2} (e^c)^2 + \frac{1}{2} \Gamma^{-1} |k_p| \bm{\tilde{\theta}}^T(t) \bm{ \tilde{\theta}}(t)\\
\dot{V}
    &= e^c \dot{e}^c + \Gamma^{-1} |k_p| \tilde{\bm{\theta}}^T \bm{\dot{ \tilde{\theta}}} = \dots\\
    &= \left( a_m + l \right) (e^c)^2\\
    & \quad     + \underbrace{e^c k_p \tilde{\bm{\theta}} \bm{ \phi}
        + \Gamma^{-1} |k_p| \tilde{\bm{\theta}}^T \bm{\dot{ \tilde{\theta}}}}_{ \overset{!}{=} 0}\\
\dot{V} &= \left( a_m + l \right) (e^c)^2 \leq 0, \qquad l < 0
\end{alignat*}

Adaptive law
\begin{alignat*}{3}
\dot{\bm{\theta}} &= -\Gamma \sign{k_p} e^c \bm{\phi}
\end{alignat*}

\paragraph{Proof} as before.
$e^c(t) \rightarrow 0$ for $t \rightarrow \infty$ %
\footnote{We assume here that $e^c(t) \rightarrow 0$ follows from
$e^o(t) \rightarrow 0$. In actuality, though, $e^o(t)$ can't
be proven for special functions. However, these cases
are usually not relevant to engineering/industry.
Therefore, \redtext{strictly speaking}, we can't actually assume
that $e^c(t) \rightarrow 0$}.

\paragraph{Questions}
\begin{itemize}
\item How do we show increased performance?\\
    (using $\| e^c(t) \|_{ \mathcal{L}_2}$ as a performance criterion)
\item How do we show that the oscillations decrease?
\end{itemize}



\subsection{Analysing transient performance}
Check the performance criterion $ \mathcal{L}_2$-norm of $e^c$
\begin{alignat*}{3}
\int_0^\infty \dot{V}(e^c, \bm{\theta}) d\tau &= V(\infty) - V(0)\\
-| a_m + l |\int_0^\infty {e^c}^2 d\tau &= V(\infty) - V(0)\\
V(0) &= \underbrace{V(\infty)}_{\geq_0} + |a_m + l| \cdot \| e^c \|_2^2\\% \int_0^\infty (e^c)^2 d\tau\\
V(0) &\geq |a_m + l| \cdot \| e^c \|_2^2 \\
\| e^c \|_2 &\leq \sqrt{\dfrac{V(0)}{|a_m + l|}}
\end{alignat*}

\begin{conclusion}{darkgreen}
\begin{alignat}{3}
\| e^c \|_2^2 &\leq \frac{1}{2} \dfrac{(e^c(0))^2 + \frac{|k_p|}{\gamma} \bm{\theta}^T(0) \bm{\theta}(0)}{|a_m + l|} 
    \label{eq:ec-2norm}
\end{alignat}
\end{conclusion}



% % \subsection{Lemma: ``Adaptive laws based on Lyapunov for
% % relative degree 1 plants''}
% % 
% % Consider the dynamical system below%
% % \footnote{think of the error dynamics, not of the plant!}.
% % \begin{alignat*}{3}
% % \dot{x}(t) &= A x(t) + b \theta^T(t) \phi(t) \quad & x &\in \R^n\\
% % y(t) &= c^Tx(t) \quad           & y, z_1 &\in \R^1\\
% % z_1(t) &= ky(t)             & \phi, \theta & \in \R^k
% % \end{alignat*}
% % ``$z_1$ allows change of symbol.\\
% % 
% % where $(A,b)$ is stabilisable and $(c^T,A)$ 
% % is detectable and $c^T(sI-A)^{-1}b \corresponds H(s)$ 
% % is SPR.\\
% %  
% % Let $\theta(t)$ be a vector of adjustable parameters.\\
% % Let $\phi(t)$ and $z_1(t)$ be time-varying functions that can
% % be measured.\\
% % 
% % Then, if $\theta(t)$ is adjusted as
% % \begin{alignat}{3}
% % \dot{\theta}(t) &= -\sign{k} z_1(t) \phi(t)
% % \end{alignat}
% % (Adaptive law for multistate systems)
% % 
% % $\Rightarrow$ the equilibrium state $(x=0, \theta=0)$ 
% % is uniformly stable in the large.\\
% % uniformly stable: not dependent on time nor on IC\\
% % in the large: IC don't matter everywhere in $\R^n$.
% % 
% % \paragraph{Proof}
% % Since $H(s)$ is SPR, it follows from KY lemma that $\exists P = P^T >0$ 
% % such that
% % \begin{alignat}{3}
% % A^TP + PA = -Q, \quad \txt{with } Q^T = Q>0\\
% % Pb &= c
% % \end{alignat}
% % 
% % Let $V$ be a positive definite function
% % \begin{alignat}{3}
% % V &= x^TPx + \frac{1}{|k|} \theta^T \theta
% % \end{alignat}
% % 
% % Its time derivative along the system is
% % \begin{alignat}{3}
% % \dot{V} &= x^T \left( PA + A^TP \right)x + 2 x^TPb \theta^T\phi - 2\theta^Ty\phi\\
% %         &= -x^TQx \leq 0
% % \end{alignat}
% % 
% % Therefore, the origin of (1), (2) is stable.
% % %% TODO correct references.
% % 
% % \paragraph{Discussion}
% % We now have a simple tool for finding adaptive laws with
% % error dynamics
% % \begin{alignat*}{3}
% % e &= \frac{1}{k^*} M(s) [ \tilde{\theta}^T \phi]
% % \end{alignat*}
% % (where $M(s)$ SPR(s)tabilisable and detectable)
% % and the adaptive laws are
% % \begin{alignat}{3}
% % \dot{\theta} &= -\sign{\varepsilon} e \Gamma \phi
% % \end{alignat}
% % 
% % If $M(s)$ has relative degree 1, it is obvious that
% % $e(t) \rightarrow 0$ for $t \rightarrow \infty$.
% % (Why?)
% % 
% % \subsection{Performance considerations}
% % Engineering criteria:\\
% % \begin{tabu} to \columnwidth{|c|c|}
% % \toprule
% % Performance & Noise\\
% % \midrule
% % Disturbance & Robustness
% % \bottomrule
% % \end{tabu}
% % 
% % \begin{itemize}
% % \item Increasing $\gamma$, we are unhappy with the oscillations of our 
% %     parameters $\theta(t)$ and therefore with the
% %     oscillations of $u(t)$.
% % \item We have no clue what the adaptive closed loop will do between
% %     $t=0$ and $t=\infty$ other than boundedness.
% % \end{itemize}
% % 
% % \paragraph{Now}
% % Deal with transient response.
% % 
% % \paragraph{Idea}
% % Adaptation changes with signals.
% % \begin{alignat*}{3}
% % \left( \dot{\theta}(t) = - \sign{\varepsilon} e \gamma \phi \right),
% % \quad e, \gamma \txt{ changeable}
% % \end{alignat*}
% % 
% % We can alter the transient with $\gamma$ (leads to oscillations),
% % or we can change $e(t)$.
% % 
% % \paragraph{So far} Open loop reference model (ORM)
% % \begin{alignat}{3}
% % \dot{x}_m(t)^o &= a_m x_m(t)^o + k_m r(t)\\
% % \end{alignat}
% % 
% % %% TODO control block
% % 
% % \paragraph{Now} Closed loop reference model (CRM)
% % \begin{alignat}{3}
% % \dot{x}_m(t)^c &= a_m x_m(t)^c + k_m r(t) - l e^c(t)
% % \end{alignat}
% % 
% % ``CRM is observer-like; $M$ helps $G$ by moving towards
% % $G$ and retreating to original position.''
% % 
% % ORM $=$ CRM if $l=0$.
