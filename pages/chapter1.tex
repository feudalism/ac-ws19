\section{Linear SISO plant}
\input{./pages/lin-control-problem}~

Solutions for $u(t)$ using
\begin{itemize}
\item Model reference control (MRC)
\item Model reference adaptive control (MRAC)
\end{itemize}

\subsection{Model reference control (MRC)}
The \greentext{plant parameters} are assumed to be known.
\begin{alignat*}{3}
    G&:~    & \dot{x}_p(t) &= \knownVar{a_p} x_p(t)
                    + \knownVar{k_p} \unknownVar{u(t)},
            \tag{\ref{eq:G-linear}}\\
            &&& \txt{IC } x_p(0) \in \R
\end{alignat*}~

Pick $u(t)$ such that the dynamical behaviour
of the closed loop is equal to that of the model.
This is done by comparing \eqn{eq:G-linear} to \eqn{eq:G-des}.
\begin{align*}
u^*(t)
    &= \frac{1}{k_p} \left( -a_p x_p(t) + a_m x_p(t) + k_m r(t) \right)\\
    &= \underbrace{\frac{a_m - a_p}{k_p}}_{a^*} x_p(t)
        + \underbrace{\frac{k_m}{k_p}}_{k^*} r(t)\\
%    &= a^* x_p(t) + k^* r(t) \\
    &= \matr{a^* & k^*} \matr{x_p(t) \\ r(t)}\\
u^*(t)
    &= \bm{\theta}^{*T} \bm{\phi}(t)
%    \numberthis \label{eq:u-perfect-linear}
\end{align*}~

Using this input%
    \footnote{%
        The starred variables with $*$ superscripts represent
        the ideal values of the control parameters.
    }%
, now the dynamics of the plant $G$ matches the dynamics
of the model $M$, as in equation \eqn{eq:G-des}.\\

However, even though now the dynamics are the same,
the initial conditions are not necessarily the same.
Would this input also work for $x_m(0) \neq x_p(0)$?
I.e., does this guarantee that
$x_p(t) \rightarrow x_m(t)$ for $t \rightarrow \infty$?

\paragraph{Dependence on the initial conditions}
To check this, we examine the error dynamics
and see if the error asymptotically goes to zero.
\begin{align*}
e(t) &= x_p(t) - x_m(t)\\
\dot{e}(t) &= \dot{x}_p(t) - \dot{x}_m(t)\\
\end{align*}

Using eqns. \eqn{eq:M} and \eqn{eq:G-des}:
\begin{align}
\dot{e}(t)    &= a_m e(t)
\label{eq:error-dynamics}
\end{align}~

\begin{conclusion}{darkgreen}
If $a_m <0$, the error dynamics are stable.
That is, $e(t) \rightarrow 0$ for any ICs.
\end{conclusion}

\paragraph{Conclusion}
\begin{itemize}
\item Model reference control (MRC) works with the error dynamics of the
    reference model
\item \redtext{We need to know all plant parameters very well} \\
    $\Rightarrow$ Problem: \redtext{uncertainty in parameters}
\end{itemize}


\subsection{Model reference adaptive control (MRAC)}
The \redtext{plant parameters} are unknown.
We assume $k_p > 0$.
\begin{alignat*}{3}
    G&:~    & \dot{x}_p(t) &= \unknownVar{a_p} x_p(t)
                    + \unknownVar{k_p} \unknownVar{u(t)},
            \tag{\ref{eq:G-linear}}\\
            &&& \txt{IC } x_p(0) \in \R
\end{alignat*}

\paragraph{Control law}
We search for (learn) the value of $\theta$ and $k$,
which are therefore functions of time.
\begin{alignat*}{3}
u(t)    %&= a(t) x_p(t) + k(t) r(t)\\
        &= \matr{a(t) & k(t)} \matr{x_p(t) \\ r(t)}\\
        &= \bm{\theta}^T(t)  \bm{\phi}(t)
\numberthis \label{eq:u-linear}
\end{alignat*}

\paragraph{Adaptive law}
Adapt the control parameters in the following fashion.
\begin{alignat*}{3}
\matr{\dot{a}(t) \\ \dot{k}(t)}
    &= - \sgn{k_p} e(t)
            \matr{\gamma_1 & 0\\
                  0 & \gamma_2} \matr{x_p(t) \\ r(t)}\\
\Rightarrow~ \dot{\bm{\theta}} &= - \sgn{k_p} e(t) \bm{\Gamma} \bm{\phi}(t)
    \numberthis \label{eq:adaptive-law}
\end{alignat*}~

The equations in \eqn{eq:adaptive-law} are nonlinear ODEs.

\paragraph{Questions}
\begin{itemize}
\item Is the closed loop stable?
\item Does, with this, $e(t) \rightarrow 0$?
\item Are the parameters $\bm{\theta}(t)$ finite? 
\item Are the parameters $\bm{\theta}(t)$ constant for $t \rightarrow \infty$?
\item Do the parameters $\bm{\theta}(t)$
        approach their `ideal' values $\bm{\theta}^*$
        for $t \rightarrow \infty$?
\end{itemize}




\section{Nonlinear SISO plant}
The linear SISO plant in the previous chapter is generalised to
a nonlinear SISO plant.\\

\begin{problem}{Control problem}
\paragraph{Given} plant $G$ and reference model $M$.
\begin{alignat*}{3}
G&:~ &
    \dot{x}_p(t) &= \unknownVar{a_p} x_p(t) + \unknownVar{k_p} u(t) \\
          &&& \quad \quad + \unknownVar{\alpha_p} f(\knownVar{z})
    \numberthis \label{eq:G-nonlinear}\\
M&:~ &
    \dot{x}_m(t) &= \knownVar{a_m} x_m(t) + \knownVar{k_m} r(t)
    \tag{\ref{eq:M}}
\end{alignat*}

\begin{itemize}
\item \redtext{$a_p$, $k_p$, $\alpha_p$} are unknown but constant
\item $f(z)$ is a nonlinear (external) function
\item $\sgn{k_p}$, $f(.)$ are known ($\knownVar{z}$ is a known signal)
\item $\alpha_m f(z)$ is not necessary
\end{itemize}

\paragraph{Goal}
$x_p(t) \rightarrow x_m(t)$ for $t \rightarrow \infty$.
\end{problem}




\subsection{Control structures}
\paragraph{Ideal control structure} based on MRC
\begin{alignat*}{3}
u^*(t)
    &= \frac{1}{k_p} \left( -a_p x_p(t) + a_m x_p(t) \right.\\
            & \left. \qquad \qquad + k_m r(t) - \alpha_p f(z) \right)\\
    &= \underbrace{\frac{\left( a_m - a_p \right)}{k_p}}_{a^*} x_p
        + \underbrace{\frac{k_m}{k_p}}_{k^*} r(t)
        + \underbrace{\frac{-\alpha_p}{k_p}}_{\alpha^*} f(z)\\
    &= \matr{a^* & k^* & \alpha^*} \matr{x_p(t) \\ r(t) \\ f(z)}\\
u^*(t)
    &= \bm{\theta}^{*T} \bm{\phi}(t)
    \numberthis \label{eq:u-ideal-nonlinear}
\end{alignat*}

\paragraph{Control law} using MRAC
\begin{alignat*}{3}
u(t) %&= a(t) x_p(t) + k(t) r(t) + \alpha(t) f(z) \\
     &= \unknownVar{\matr{a(t) & k(t) & \alpha(t)}} \matr{x_p(t) \\ r(t) \\ f(z)}\\
u(t) &= \bm{\unknownVar{\theta}}^T(t) \bm{\phi}(t)
    \numberthis \label{eq:u-nonlinear}
\end{alignat*}
\redtext{$a(t),~ k(t),~ \alpha(t)$} unknown.




\subsection{Error dynamics}
In adaptive control, the current estimated parameters are varying.
We therefore have
the following error in parameters
as deviations from the unknown but ideal and constant
real parameters:
\begin{alignat*}{3}
\left.
\begin{array}{rl}
    \tilde{a}(t) &= a(t) - a^*\\
    \tilde{k}(t) &= k(t) - k^*\\
    \tilde{\alpha}(t) &= \alpha(t) - \alpha^*
\end{array}
\right\rbrace \tilde{\bm{\theta}}(t) = \bm{\theta}(t) - \bm{\theta}^*
\end{alignat*}%
%
\begin{alignat*}{3}
\dot{e}(t) &= \dot{x}_p(t) - \dot{x}_m(t)\\
        &= a_p x_p(t)\\
        & \qquad + k_p \left( a(t) x_p(t) + k(t) r(t) + \alpha(t) f(z) \right)\\
        & \qquad + \alpha_p f(z)\\
        & \qquad - \left( a_m x_m(t) + k_m r(t) \right)\\
    &= a_p x_p(t) - a_m x_m(t)\\
        & \qquad + k_p a(t) x_p(t)\\
        & \qquad + k_p \underbrace{ \left(  k(t) - \frac{k_m}{k_p} \right) }_{ \tilde{k}(t)} r(t)\\
        & \qquad + k_p \underbrace{\left( \alpha(t) - \frac{\alpha_p}{k_p} \right)}_{\tilde{\alpha}(t)} f(z)\\
    &= \underbrace{\left( a_m - k_p a^* \right)}_{a_p} x_p(t)
        - a_m x_m(t) \\
        & \qquad + k_p a(t) x_p(t) \\
        & \qquad + k_p \tilde{k}(t) r(t)
        + k_p \tilde{\alpha}(t) f(z)\\
    &= \knownVar{a_m} e(t)
        + k_p \tilde{a}(t) x_p(t)\\
        & \qquad + k_p \tilde{k}(t) r(t)
        + k_p \tilde{\alpha}(t) f(z)
\end{alignat*}
\begin{alignat*}{3}
\dot{e}(t) &= \knownVar{a_m} \measuredVar{e(t)}
        + \unknownVar{k_p \matr{\tilde{a}(t) & \tilde{k(t)} & \tilde{\alpha}(t)}}
        \measuredVar{\matr{x_p(t) \\ r(t) \\ f(z)}} \\
\dot{e}(t)
    &= \knownVar{a_m} \measuredVar{e(t)}
        + \frac{1}{ \unknownVar{k^*}} \knownVar{k_m} \unknownVar{\bm{\tilde{\theta}}^T(t)} \measuredVar{\bm{\phi}(t)} 
    \numberthis \label{eq:error-dynamics-nonlinear}
\end{alignat*}~

The error dynamics can be rewritten using an operator
$ \knownVar{M(s)} = \dfrac{ \knownVar{k_m}}{s - \knownVar{a_m}}$,
which is non other than the transfer function of the
reference model!
\begin{alignat*}{3}
\dot{e}(t)
    &= \knownVar{a_m} \measuredVar{e(t)}
        + \unknownVar{\frac{1}{k^*} k_m \bm{\theta}^T(t)} \measuredVar{\bm{\phi}(t)} \\
\left( s - \knownVar{a_m} \right) e(t)
    &= \frac{1}{ \unknownVar{k^*}} \knownVar{k_m} \unknownVar{\bm{\theta}^T(t)} \measuredVar{\bm{\phi}(t)} \\
e(t)    &= \frac{1}{ \unknownVar{k^*}} \knownVar{M(s)} \unknownVar{\bm{\theta}^T(t)} \measuredVar{\bm{\phi}(t)}
    \tag{\ref{eq:error-dynamics-nonlinear}}
\end{alignat*}

All unknown parameters appear linearly (affine,
`linear in the parameters').
The error dynamics \eqn{eq:error-dynamics-nonlinear} is a nonlinear differential equation.
When is it stable? $ \rightarrow$ Lyapunov.



\subsection{Lyapunov-like function}
\paragraph{New interpretation of Lyapunov}
Nothing to do with energy.
$\bm{V}$ affects the scaling of the distance of $\bm{x}$ 
from the origin in the phase portrait.
\begin{align*}
\norm{\bm{x}}^2_{\bm{V}} = \bm{x}^T \bm{V} \bm{x}, \qquad \bm{V} \succ 0
\end{align*}~

\begin{conclusion}{darkgreen}
\textbf{All Lyapunov says is}: how far is $\bm{x}$ 
from the origin? We want to find some type of measure
for that.
\end{conclusion}

\paragraph{Lyapunov function} (Lyapunov-like)\\
We want the output error $e(t)$ as well as the parameter
error $\bm{ \tilde{\theta}}(t)$ to go to zero.
$\bm{\Gamma} \succ 0$ symmetrical, positive definite.
\begin{alignat*}{3}
V(e, \tilde{\bm{\theta}})
    &= \frac{1}{2}e^2 + \frac{1}{2}|k_p| \left( \tilde{\bm{\theta}}^T \bm{\Gamma}^{-1} \tilde{\bm{\theta}} \right)
    \numberthis \label{eq:V}\\
\dot{V}
    &= e \dot{e} + \frac{1}{2}|k_p|
    \undertxt{\left( 2 \tilde{\bm{\theta}}^T \bm{\Gamma}^{-1} \dot{\tilde{\bm{\theta}}} \right)}{\footnotemark}
\end{alignat*}
\footnotetext{Possible due to $\bm{\Gamma}$ symmetrical.}

Substitute $\dot{e}$ using equation \eqn{eq:error-dynamics-nonlinear}.
\begin{alignat*}{3}
\dot{V}  &= a_me^2 + e k_p \tilde{\bm{\theta}}^T \bm{\phi}
        + \frac{1}{2} |k_p| \left( 2 \tilde{\bm{\theta}^T \bm{\Gamma}^{-1} \dot{\tilde{\bm{\theta}}}} \right)\\
    &= a_me^2 + |k_p| \tilde{\bm{\theta}}^T
        \underbrace{\left(
        \sgn{k_p} e \phi + \bm{\Gamma}^{-1} \dot{\tilde{\theta}}\right)}_{
        \overset{!}{=} 0}
\end{alignat*}

The second term is set to 0, because we want $V \preceq 0$ and
we don't know all the signs of the terms.
This will define the \textbf{adaptive law}.
\begin{alignat*}{3}
\dot{\tilde{\theta}}(t) &= - \sgn{k_p} \bm{\Gamma} \bm{\phi}(t) e(t)\\
\dot{\theta}(t) &= - \sgn{k_p} \bm{\Gamma} \bm{\phi}(t) e(t)\\
\end{alignat*}

With the adaptive law, we obtain for $\dot{V}$:
\begin{alignat}{3}
\dot{V} &= a_m e^2 \preceq 0 \label{eq:Vdot}
\end{alignat}

\paragraph{Remark}
$e(t)$ does not have to be 0 -- why?
% \begin{itemize}
% \item We do not need $\dot{V}$ to approach zero.
% \item $\dot{V}=0$ does not imply that $V$ has a limit as $t \rightarrow \infty$.
%     (although this is known, see footnote below%
%     \footnote{A function $V$ that is bounded from below $V \succeq 0$
%         and \textbf{non-increasing} $\dot{V} \preceq 0$
%         has a limit as $t \rightarrow \infty$}).
% \item In other words, $\dot{V}=0$ does not imply that the errors go to zero,
%     and vice versa.
% \end{itemize}~  

\begin{conclusion}{darkred}
If the derivative of a function $\rightarrow 0$,
that \redtext{does not} imply that the function
has a limit, and vice versa:
if a function has a limit, that doesn't mean its
derivative $\rightarrow 0$.
\begin{alignat*}{3}
\lim_{t \rightarrow \infty} \dot{f}(t) = 0
    \nLeftrightarrow \lim_{t \rightarrow \infty} f(t) = k
\end{alignat*}
\end{conclusion}~

Counterexamples:
\begin{align*}
f(t) &= \sin \left( \ln t \right)\\
\nexists \lim_{t \rightarrow \infty} f(t), \quad
\dot{f}(t) &= \frac{\cos \left( \ln t \right)}{t} \rightarrow 0\\
~\\
f(t) &= e^{-t} \sin (e^{2t})\\
 \lim_{t \rightarrow \infty} f(t) &= 0\\
\dot{f}(t) &= -e^{-t}\sin(e^{2t}) + e^t \sin(e^{2t}) \\
    & \quad \rightarrow    \txt{ explodes!}
\end{align*}

\paragraph{Are the error dynamics stable?}
\begin{itemize}
\item Measure (some of) the states
\item Apply $V = f(e, \tilde{\bm{\theta}})$
\item $V \rightarrow \infty$? Or $V \downarrow$?\\
    $\Rightarrow$ analyse time derivative $\dot{V}$
\item If we show $\dot{V} \rightarrow 0$, then $e \rightarrow 0$.
\end{itemize}

\paragraph{Extensions to Lyapunov}
There are two well-known extensions to Lyapunov to prove
asymptotic stability, even if $\dot{V} \preceq 0$.
\begin{enumerate}
\item LaSalle's invariance principle
    (\redtext{only for autonomous systems})
\item Barbalat's lemma
    (\greentext{OK for non-autonomous systems})
\end{enumerate}~

Our system's error dynamics are non-autonomous,
$\dot{e} = f(t, \dots)$, due to
following another system (Figure \ref{fig:non-autonomous-system}).
\begin{figure}[H]
\centering
\inkscape[\normalsize]{e-non-auto}{0.25}
\caption{Non-autonomous dynamics}
\label{fig:non-autonomous-system}
\end{figure}



\subsection{Closed loop stability analysis}
Using Barbalat's Lemma (Variant A)%
\footnote{Chapter \vref{ch:barbalat}}
 on the function $V$,
we need to fulfill the following conditions:
\begin{enumerate}[label=(\roman*)]
\item $V$ is differentiable\\
    \greentext{Yes, $\exists~ \dot{V} = a_m e^2(t)$}
\item $V$ has a finite limit as $t \rightarrow \infty$ \\
    \greentext{Yes, as $V \succeq 0$ and  $\dot{V} \preceq 0$}
\item $\dot{V}$ is uniformly continuous $\Leftarrow \exists ~ \ddot{V}$
    (sufficient condition%
    \footnote{Chapter \vref{ch:uniform-continuous}}    
    )\\
    \redtext{Is $\ddot{V} = 2 a_m e \dot{e}$ bounded?}
\end{enumerate}~

Boundedness of $e(t)$
\begin{itemize}
\item As $V$ is bounded from below and non-increasing,
    $V$ has a limit as $t \rightarrow \infty$.
\item Tracking error $e(t)$ and parameter errors $ \tilde{\bm{\theta}}(t)$
    are bounded.
\item As $\tilde{\bm{\theta}}(t)$ bounded and $\bm{\theta}^*$ constant,
    $\bm{\theta}(t)$ is bounded.
\end{itemize}~

Boundedness of $\dot{e}(t)$ 
\begin{itemize}
\item \redtext{Assume $r(t)$ bounded}%
    \footnote{Reasonable assumption, because why would we want
    to use an unbounded input?}, then, from the reference
    model equation \eqn{eq:M}:\\
    $x_m(t), \dot{x}_m(t)$ bounded ($\because~ M$ is stable)
\item $x_p(t) = \bounded{e(t)} + \bounded{x_m(t)}$\\
    $\Rightarrow x_p(t)$ bounded.
\item $u(t) = \bm{\theta}^T(t) \bm{\phi}(t)$ bounded if $\bm{\phi}(t)$ bounded.\\
    $\bm{\phi}(t) = \matr{\bounded{r(t)} & \bounded{x_p(t)} & \bounded{f(z)}}^T$\\
    (new requirement: \redtext{$f(z)$ needs to be bounded}).
\item $\dot{x}_p(t) = a_p \bounded{x_p(t)} + k_p \bounded{u(t)}$\\
    $\Rightarrow \dot{x}_p(t)$ bounded
\item $\dot{e}(t) = \bounded{\dot{x}_p(t)} - \bounded{\dot{x}_m(t)}$
    is bounded.
\end{itemize}~

All the conditions of Barbalat's lemma thus fulfilled,
we can conclude that the derivative of $V$ approaches zero
for $t \rightarrow \infty$.
\begin{alignat*}{3}
& ~ & \lim_{t \rightarrow \infty} \dot{V} &= 0\\
\Rightarrow& ~ & \lim_{t \rightarrow \infty} e(t) &= 0
\end{alignat*}



\subsection{Nonlinearities}
\begin{thrm}{Nonlinear SISO plant}
\paragraph{Given} plant $G$ and reference model $M$.
\begin{alignat*}{3}
G&:~ &
    \dot{x}_p(t) &= \unknownVar{a_p} x_p(t) + \unknownVar{k_p} u(t) \\
          &&& \quad \quad + \unknownVar{\alpha_p} f(\knownVar{z})
    \tag{\ref{eq:G-nonlinear}}\\
M&:~ &
    \dot{x}_m(t) &= \knownVar{a_m} x_m(t) + \knownVar{k_m} r(t)
    \tag{\ref{eq:M}}
\end{alignat*}

The input
\begin{alignat*}{3}
u(t) &= \bm{\theta}^T(t) \bm{\phi}(t) \tag{\ref{eq:u-nonlinear}}\\
    \tag{\ref{eq:adaptive-law}}
    \qquad \txt{with } \dot{ \bm{\theta}} &= -\sgn{k_p} \bm{\Gamma} \bm{\phi} e\\
    \qquad \txt{and } \bm{\phi}(t) &= \matr{r(t) & x_p(t) & f(z)}^T
\end{alignat*}
renders the origin asymptotically stable and guarantees
$x_p(t) \rightarrow x_m(t)$ as $t \rightarrow \infty$.
\end{thrm}~



We can add, arbitrarily, many `nonlinearities' $f_i(z_j)$ with
unknown gains $\alpha_i$. The nonlinearity functions
\greentext{need not be continuous}.
The only requirement:
\begin{conclusion}{darkred}
\begin{alignat*}{3}
f_i(z_j) \in \mathcal{L}_\infty
\end{alignat*}
Nonlinearities are bounded at all times%
    \footnote{For the boundedness of $u(t)$ and therefore $\dot{e}(t)$}.
\end{conclusion}

\paragraph{The function $z(t)$} $z$ is a placeholder. $z(t)$
can be an external or an internal signal.
\begin{figure}[H]
    \centering
    \tikzsetnextfilename{nonlinearities}
    {\footnotesize
\begin{tikzpicture}
\def\w{0.5cm}
\def\h{1cm}

\node (kp) [block] {$k_p$};
\node (comp1) [comp, right=\w of kp] {};
\node (comp2) [comp, right=\w of comp1] {};
\node (i) [block,right=1.5*\w of comp2] {$\int$} {};
\node (cout) [cout, right=1.5*\w of i] {};

\node (ap) [block] at ($($(comp1)!0.5!(cout)$) + (0,0.7*\h)$) {$a_p$};

\node (f2) [nonlin] at ($($(comp2)!0.7!(cout)$)+(0,-\h)$) {$f_2$};
\node (alp2) [block, left=0.8*\w of f2.west] {$\alpha_2$};

\node(alp1) [block, below=0.5*\h of comp1, anchor=north] {$\alpha_1$};
\node (f1) [nonlin, below=0.5*\h of alp1.south, anchor=north, text width=0.6cm] {$f_1(.)$};

\begin{scope}[arrow]
\draw ($(kp.west) + (-\w,0)$) -- (kp) node [pos=0.25,above] {$u$};
\draw (kp) -- (comp1);
\draw (comp1) -- (comp2);
\draw (comp2) -- (i) node [pos=0.5,above] {$ \dot{x}_p$};
\draw (i) -- (cout) -- ++(\w,0) node [pos=1,above] {$x_p$};

\draw (cout) |- (ap);
\draw (ap) -| (comp1);

\draw (cout) |- (f2);
\draw (f2) -- (alp2);
\draw (alp2) -| (comp2);

\draw ($(f1.south)+(0,-0.5*\h)$) -- (f1) node [pos=0.2,right] {$d$ (disturbances)};
\draw (f1) -- (alp1);
\draw (alp1) -- (comp1);
\end{scope}
\end{tikzpicture}
}

    \caption{Plant $G$ with nonlinearities}
    \label{fig:nonlinearities}
\end{figure}

\paragraph{Questions}
\begin{itemize}
\item Can we do $f(u)$?\\
    Possible, but \redtext{solving for $u = \cdots f(u) \cdots$ is difficult}.
\item Can $f(.)$ be a differential operator (filter)?\\
    \greentext{Yes}. If the filter is linear, then a solution
    definitely exists.
    \begin{figure}[H]
        \centering
        \tikzsetnextfilename{nonlin-filter}
        {\footnotesize
\begin{tikzpicture}
\def\w{0.5cm}
\def\h{1cm}

\node (start) {};
\node (coutu) [cout, right=0.5*\w of start] {};
\node (comp1) [comp, right=5*\w of coutu] {};
\node (kp) [block] at ($(coutu)!0.5!(comp1)$) {$k_p$};
\node (cout) [cout, right=5*\w of comp1] {};
\node (i) [block] at ($(comp1)!0.5!(cout)$) {$\int$} {};
 
\node (ap) [block] at ($($(comp1)!0.5!(cout)$) + (0,0.7*\h)$) {$a_p$};
 
\node (fx) [nonlin] at ($($(comp1)!0.7!(cout)$)+(0,-\h)$) {$F_x$};
\node (filternote) [note,below=2mm of fx.south,anchor=center] {many filters};
\node (alp2) [block] at ($($(comp1)!0.3!(cout)$)+(0,-\h)$) {$\bm{\alpha}_x^T$};

\node (fu) [nonlin] at ($($(coutu)!0.3!(comp1)$)+(0,-\h)$) {$F_u$};
\node (alp1) [block] at ($($(coutu)!0.7!(comp1)$)+(0,-\h)$) {$\bm{\alpha}_p^T$};
 
\begin{scope}[arrow]
    \draw (start) -- (coutu)  node [pos=0.1,above] {$u$}
                  -- (kp) ;
    \draw (kp) -- (comp1);
    \draw (comp1) -- (i) node [pos=0.5,above] {$ \dot{x}_p$};
    \draw (i) -- (cout) -- ++(\w,0) node [pos=1,above] {$x_p$};
     
    \draw (cout) |- (ap);
    \draw (ap) -| (comp1);
     
    \draw (cout) |- (fx);
    \draw (alp2) -| (comp1);
     
    \draw (coutu) |- (fu);
    \draw (alp1.east) -- ++(2mm, 0) -- (comp1);
\end{scope}

\begin{scope}[vectline]
    \draw (fx) -- (alp2);
    \draw (fu) -- (alp1);
\end{scope}
\end{tikzpicture}
}

        \caption{Plant $G$ with nonlinearities as filters}
        \label{fig:nonlin-filter}
    \end{figure}
    The filters $F_u$ and  $F_x$ are stable dynamical systems.
    \redtext{If they are not stable, there is a higher chance of
    getting an infinite output $\rightarrow$ conflicts with bounded
    output requirement.}
\end{itemize}~

If no such filters are present in the plant $G$,
then the plant has an order of 1.\\

Let's say we have an \redtext{offset in the plant input} of unknown magnitude,
and the plant has otherwise \greentext{known parameters}.
\begin{alignat*}{3}
\dot{x}_p(t) &= \knownVar{a_p} x_p(t) + \knownVar{k_p} u(t) + \unknownVar{\alpha_p} \cdot 1\\
u(t) &= \knownVar{a^*} x_p(t) + \knownVar{k^*} r(t) - \frac{ \unknownVar{\alpha(t)}}{k_p}
\end{alignat*}

Closed loop becomes
\begin{alignat*}{3}
\dot{x}_p(t) &= a_m x_p(t) + k_m r(t) + \tilde{\alpha}(t)
\end{alignat*} 

Error dynamics
\begin{alignat*}{3}
\dot{e} &= a_m e(t) + \tilde{\alpha}(t)
\end{alignat*}

Lyapunov-like function
\begin{alignat*}{3}
V &= \frac{1}{2} e^2 + \frac{1}{2} \tilde{\alpha}^2\\
\dot{V} &= e \dot{e} + \tilde{\alpha} \dot{ \tilde{\alpha}} + \cdots\\
    &= a_me^2 + \tilde{\alpha} \undertxt{\left( e + \dot{ \tilde{\alpha}} \right)}{to set to 0}
\end{alignat*}

Setting the second term to zero ensures the negative
semi\-definite\-ness of $\dot{V}$.
\begin{alignat}{3}
\dot{ \tilde{\alpha}} &= -e \label{eq:integral-error}
\end{alignat}

Equation \eqn{eq:integral-error} above implies
$u = \cdots + \int e dt + \cdots$, i.e.,
that the controller contains an I-part.
As a result of this, there are no steady state errors
caused by model uncertainties.
The controller eliminates offset at input of plant.\\

Equation \eqn{eq:integral-error} is a pure integrator
acting on control error. This is a linear controller!
We have learned: integrators `learn' input offsets of
the plant and correct them.\\

\begin{conclusion}{darkgreen}
Adaptive controllers can be interpreted as nonlinear
PI controllers.
\end{conclusion}


\subsection{Uniformly continuous functions}
\label{ch:uniform-continuous}
\begin{definition}{Uniformly continuous function}
A function $f(t): \R \rightarrow \R$ 
is uniformly continous, if\\
$\forall \varepsilon>0:
    \quad \exists~ \delta = \delta(\varepsilon) >0,$
\begin{alignat*}{3}
\forall |t_2 - t_1| &\leq \delta\\
\Rightarrow |  f(t_2) - f(t_1) | &\leq \varepsilon
\end{alignat*}
%\label{ch:uniform-continuous}
\end{definition}

\begin{conclusion}{darkgreen}
\textbf{Sufficient condition for uniformly continuous functions}:
If the derivative $\dot{f}(t)$ exists (i.e. bounded),
$\Rightarrow f(t)$ is uniformly continuous.
\end{conclusion}~



\subsection{Barbalat's lemma}
\label{ch:barbalat}
\begin{lemma}{Barbalat Variant A}
    If $f(t): \R \rightarrow \R$
    \begin{enumerate}
    \item is a differentiable function, $\dot{f} \in \mathcal{L}_\infty$
    \item has a finite limit as $t \rightarrow \infty$, $f \in \mathcal{L}_\infty$ 
    \item $\dot{f}(t)$ is uniformly continuous, $\ddot{f} \in \mathcal{L}_\infty$
    \end{enumerate}
\begin{alignat*}{3}
    \Rightarrow \lim_{t \rightarrow \infty} \dot{f}(t) &= 0
\end{alignat*}
\end{lemma}

\begin{lemma}{Barbalat Variant B}
If
\begin{enumerate}
\item $f(t): \R \rightarrow \R$ is uniformly continous $\forall t$
\item $\exists~ \lim_{t \rightarrow \infty} \int_0^t f(\tau) d\tau$
\end{enumerate}
\begin{alignat*}{3}
\Rightarrow \lim_{t \rightarrow \infty} f(t) &= 0
\end{alignat*}
\end{lemma}

\begin{lemma}{Barbalat Variant C}
If
\begin{enumerate}
\item $f \in \mathcal{L}_\infty$ 
\item $\dot{f} \in \mathcal{L}_\infty $
\item $f \in \mathcal{L}_2$,
\end{enumerate}
\begin{alignat*}{3}
\Rightarrow |f(t)| \rightarrow 0 \txt{ as } t \rightarrow \infty
\end{alignat*}
\end{lemma}


\subsection{Signal norms and functional spaces}
\paragraph{Idea} quantify magnitude of a signal $x(t)$
-- ``How big is a signal?''\\

\begin{definition}{Signal norm}
    \paragraph{Given}
    \begin{alignat*}{3}
    x(t): \R^+ \rightarrow \R^n, \quad  \quad \R^+ = [0, \infty)
    \end{alignat*}

    \paragraph{p-Norm}
        \begin{alignat*}{3}
        \|x_p\| &= \left( \int_0^\infty |x(t)|^p dt \right)^{1/p}
            & \qquad p \in (0, \infty)
        \end{alignat*}
\end{definition}~

\paragraph{Distance} vector 2-norm of $\bm{x}(t)$,
i.e. $|\bm{x}|$.

\paragraph{Max. value}
``When power $\infty$, only the greatest value survives''

\begin{alignat*}{3}
\|x\|_\infty &= \sup_{t \in \R^+} |x(t)|\\
    & \corresponds \txt{ highest value of } x (t).
\end{alignat*}

\begin{definition}{Functional space}
\begin{alignat*}{3}
\mathcal{L}_p = \{x(t) \in \R^n : \undertxt{\| x \|_p < \infty}{exists} \}
\end{alignat*}
\end{definition}~

$x(t) \in \mathcal{L}_p$
\begin{itemize}
\item $x$ is bounded
\item ``$x$'s highest value exists and is not infinity.''
\end{itemize}~ \\

\begin{example}{Functional space}
Show that $e \in \mathcal{L}_\infty$ \\
  $e$ is in $V$ and $V$ is bounded, $\therefore e \in \mathcal{L}_\infty$.
\end{example}

\begin{example}{Functional space}
Show that $e \in \mathcal{L}_2$ 
  \begin{alignat*}{3}
      \int_0^\infty \dot{V} dt &= V(\infty) - V(0), \qquad \txt{is bounded}.\\
      a_m \int_0^\infty e^2 dt &\in \mathcal{L}_\infty\\
      \| e \|^2 &\in \mathcal{L}_\infty\\
      \Rightarrow e &\in \mathcal{L}_2
  \end{alignat*}
\end{example}





\section{Positive real functions}
\begin{definition}{Positive real function 1}
A rational function $H(s): \Z \rightarrow \Z$, $s = \sigma + j\omega$ is
positive real (PR), if
\begin{enumerate}[label=(\roman*)]
\item $H(s)$ is real for real $s$
\item  $\Re\{H(s)\} \geq 0$ for  $\Re\{s\} > 0$
\end{enumerate}
\end{definition}~

\begin{definition}{Positive real function 2}
A rational function $H(s)$ is positive real, if
\begin{enumerate}[label=(\roman*)]
\item $H(s)$ is analytic
%\footnote{\textbf{analytic}: ``converges to Taylor series, somewhat smooth.''}
 in $\Re\{s\} >0$ 
    \begin{itemize}
    \item $H(s)$ has no poles in RHP ($\Re \set{s} > 0$)
    \item $H(s)$ is stable
    \end{itemize}
\item $\Re \set{H(j\omega)} \geq 0, \quad \forall\omega \in [0, \infty]$    
    \begin{itemize}
        \item Nyquist of $H(s)$ is in the RHP
        \item phase $\angle H(j\omega) \in [-90^{\circ}, +90^{\circ}]$
        \item rel. degree of $H(s)$ is 0 or 1
    \end{itemize}
\item any pure imaginary pole $j\omega$ of $H(s)$ is
    a simple pole, and the residue
\begin{alignat*}{3}
\lim_{s \rightarrow j\omega} \left( s - j\omega \right) H(s)
\end{alignat*}
    is positive semidefinite.\\

    Alternatively: $H(\infty) > 0$ or 
 \begin{alignat*}{3}
\lim_{\omega \rightarrow \infty} \omega^2 \Re \set{H(j\omega)} \geq 0
\end{alignat*}
\end{enumerate}
\end{definition}~

\begin{definition}{Strictly positive real functions}
$H(s)$ is strictly positive real (SPR) if $H(s - \varepsilon)$ 
is PR for some $\varepsilon > 0$.
\end{definition}~

\begin{definition}{SPR lemma}
$H(s)$ is SPR if
\begin{enumerate}[label=(\roman*)]
\item $H(s)$ is Hurwitz\\
    all poles on LHP, none are purely imaginary
\item $\Re \set{H(j\omega)} > 0, \quad \forall \omega \in \R$\\
    phase $\in (-90 ^{\circ}, 90 ^{\circ})$\\
    rel. degree $\in \set{0, 1}$
\item $H( \infty) > 0$ (positive gain for proper $H$) or 
\begin{alignat*}{3}
\lim_{\omega \rightarrow \infty} \omega^2 \Re \set{H(j\omega)} > 0
\end{alignat*}
positive gain for relative degree 1
\end{enumerate}
\end{definition}

\paragraph{Note} relative degree of a system corresponds to
its response delay.

\paragraph{Discussion}
If $H(s)$ is SPR, then so is the inverse $H^{-1}(s)$
(stable poles, stable zeroes).\\

SPR  $\rightarrow H(s)$ is stable, minimal-phase.
I.e., only stable zerioes, because zeroes are in LHP.
\begin{alignat*}{3}
\txt{rel. degr.} &\leq 1 \qquad(-1, 0, 1)\\
\angle H(j\omega) &\in (-90 ^{\circ}, 90 ^{\circ})\\
     & \quad \txt{ with positive gain } \forall \omega
\end{alignat*}


\section{Kalman-Yakobovich lemma (KY)}
``Maier version; how to design a controller given SPR''\\

\begin{lemma}{Kalman-Yakobovich Lemma}
    \paragraph{Given}
    \begin{itemize}
    \item a scalar $\gamma \geq 0$
    \item vectors $\bm{b}$ and $\bm{c}$,
    \item an asymptotically stable matrix $\bm{A}$%
        \footnote{positive eigenvalues}, and
    \item a positive definite matrix $\bm{L} \succ 0$,
    \end{itemize}
    \begin{alignat*}{3}
    \txt{If } H(s) &\corresponds \frac{1}{2} \gamma + \bm{c}^T (s\bm{I}-\bm{A})^{-1} \bm{b}\\
        \Rightarrow H(s) &\txt{ is SPR}
    \end{alignat*}~

    Then, there exist
    \begin{itemize}
    \item a scalar $\varepsilon >0$
    \item a vector $\bm{q}$, and
    \item a symmetric positive definite matrix $\bm{P}$,
    \end{itemize}
    s.t.
    \begin{alignat}{3}
    \bm{A^TP + PA}   &= -\bm{qq}^T - \varepsilon \bm{L}\\
    \bm{Pb} - \bm{c}      &= \sqrt{\gamma} \bm{q}
    \end{alignat}
\end{lemma}

\paragraph{Using it}
We only need $\gamma=0$ in all cases (in this course).
Hence we can say: if $H(s)$ is SPR $\Rightarrow \exists ~ \bm{P}=\bm{P}^T>0$,\\
s.t.
\begin{alignat*}{3}
\bm{A^TP + PA}   &= -\bm{Q}\\
\bm{Pb} &= \bm{c} \qquad \txt{(s. note)}\footnotemark  \\
\end{alignat*}%
\footnotetext{"boundary cond., means SPR"}%
where $\bm{Q}=\bm{Q}^T>0$

\begin{lemma}{Adaptive laws based on Lyapunov}
(For rel. degree 1 plants)\\
Consider the dynamical system below%
\footnote{This is the error dynamics, \redtext{not the plant}!}%
\footnote{$z_1$ allows change of symbol with respect to the output $y(t)$}.
\begin{alignat*}{3}
\dot{\bm{x}}(t) &= \bm{A} \bm{x}(t) + \bm{b} \bm{\theta}^T(t) \bm{\phi}(t), \quad &\bm{x} &\in \R^n\\
y(t) &= \bm{c}^T \bm{x}(t)      & y, z &\in \R^1\\
z_1(t) &= k y(t)      & \bm{\phi}, \bm{\theta} &\in \R^k
\end{alignat*}

where
\begin{itemize}
\item $(\bm{A}, \bm{b})$ is stabilisable
\item $(\bm{c}^T, \bm{A})$ is detectable
\item $\bm{c}^T \left( s \bm{I} - \bm{A} \right)^{-1} \bm{b} \corresponds H(s)$
    is SPR
\end{itemize}~

Let $\bm{\theta}(t)$ be a vector of adjustable parameters.\\
Let $\bm{\phi}(t)$ and $z_1(t)$ be time-varying functions that can be measured.\\

Then, if $\bm{\theta}(t)$ is adjusted as
\begin{alignat}{3}
\bm{\dot{\theta}}(t) &= - \sign{k} z_1(t) \bm{\phi}(t)
\label{eq:adaptive-law-multistate}
\end{alignat}

$\Rightarrow$ the \greentext{equilibrium state} $\left( \bm{x}=0, \bm{\theta}=0 \right)$ 
is \greentext{uniformly stable%
\footnote{uniformly stable: not dependent on time nor on IC}
 in the large%
\footnote{in the large: IC don't matter anywhere in $\R^n$}}.
\end{lemma}


\section{Performance considerations}
\section{Performance considerations}
Performance criteria:
\begin{center}
\begin{tabu} to \columnwidth {cc}
Performance & Noise\\
Disturbances    & Robustness
\end{tabu}
\end{center}~

\begin{itemize}
\item Increasing $\gamma$, we are unhappy with the
    \redtext{oscillations of our parameters $\bm{\theta}$}
    and therefore with the oscillations of $u(t)$.
\item We have no clue what the adaptive closed loop
    will do between $t=0$ and $t=\infty$ other than boundedness
\end{itemize}

\subsection{Adaptation with a closed loop reference model}
\paragraph{Now} Deal with transient response
\paragraph{Idea} Adaptation changes with signals
\begin{alignat*}{3}
    \dot{\theta} &= - \sgn{\varepsilon} \measuredVar{e \gamma} \phi
\end{alignat*}
where the value of $e$ and $\gamma$ are changeable.\\
$\Rightarrow$ we can alter the transient with $\gamma$ (leads to oscillations),
\greentext{or we can change $e(t)$}.

\paragraph{So far} Open loop reference model (ORM)
\begin{alignat*}{3}
\dot{x}_m^o(t) &= a_m x_m^o(t) + k_m r(t)
    \tag{\ref{eq:M}}
\end{alignat*}

\paragraph{Now} Closed loop reference model (CRM)
\begin{alignat}{3}
\dot{x}_m^c(t) &= a_m x_m^c(t) + k_m r(t) - l e^c(t)
    \label{eq:M-crm}
\end{alignat}~

ORM $=$ CRM if $l=0$.\\

``CRM is observer-like; $M$ helps $G$ by moving towards
$G$ and retreating to original position.''\\
\redtext{Through the movement, the reference model now
has a different behaviour ($M \rightarrow M'$!)
and the plant $P$ is trying to follow $M'$.}\\

\begin{variables}
\gamma  & learning effect\\
        & decreasing $\gamma$ helps $P$ follow $M'$,\\
        & but the learning becomes slower\\
l       & movement to $P$ \\
        & increasing $l$ helps $P$ follow $M'$
\end{variables}


\subsection{Stability proof}
\begin{alignat*}{3}
\dot{x}_m^c(t) &= a_m x_m^c(t) + k_m r(t) - le^c(t)
    \tag{\ref{eq:M-crm}}\\
\dot{x}_p(t) &= a_p x_p(t) + k_p u(t) 
    \tag{\ref{eq:G-linear}}
\end{alignat*}

Input is
\begin{alignat*}{3}
u &= \matr{a(t) & k(t)} \matr{x_p(t) \\ r(t)}% =  a(t) x_p(t) + k(t) r(t)\\
    = \bm{\theta}^T(t) \bm{\phi}(t) 
\end{alignat*}

\begin{alignat*}{3}
\dot{x}_p(t) &= a_m x_p(t) + k_m r(t) + k_p \bm{\theta}^T(t) \bm{\phi}(t)
\end{alignat*}

Tracking error
\begin{alignat*}{3}
\dot{e}^c(t) &= \dot{x}_p(t) - \dot{x}_m^c(t)\\
    &= \left( a_m + l \right) e^c(t) + k_p \bm{\theta}^T(t) \bm{\phi}(t)
\end{alignat*}

Lyapunov-like function
\begin{alignat*}{3}
V(e^c, \bm{\theta})
    &= \frac{1}{2} (e^c)^2 + \frac{1}{2} \Gamma^{-1} |k_p| \bm{\theta}^T(t) \bm{\theta}(t)\\
\dot{V}
    &= e^c \dot{e}^c + \Gamma^{-1} |k_p| \bm{\theta}^T \bm{\dot{\theta}} = \cdots\\
    &= \left( a_m + l \right) (e^c)^2\\
    & \quad     + \underbrace{e^c k_p \bm{\theta} \bm{ \phi}
        + \Gamma^{-1} |k_p| \bm{\theta}^T \bm{\dot{\theta}}}_{ \overset{!}{=} 0}\\
\dot{V} &= \left( a_m + l \right) (e^c)^2 \leq 0, \qquad l < 0
\end{alignat*}

Adaptive law
\begin{alignat*}{3}
\dot{\bm{\theta}} &= -\Gamma \sign{k_p} e^c \bm{\phi}
\end{alignat*}

\paragraph{Proof} as before.
$e^c(t) \rightarrow 0$ for $t \rightarrow \infty$ %
\footnote{We assume here that $e^c(t) \rightarrow 0$ follows from
$e^o(t) \rightarrow 0$. In actuality, though, $e^o(t)$ can't
be proven for special functions. However, these cases
are usually not relevant to engineering/industry.
Therefore, \redtext{strictly speaking}, we can't actually assume
that $e^c(t) \rightarrow 0$}.

\paragraph{Questions}
\begin{itemize}
\item How do we show increased performance?\\
    (using $\| e^c(t) \|_{ \mathcal{L}_2}$ as a performance criterion)
\item How do we show that the oscillations decrease?
\end{itemize}



\subsection{Analysing transient performance}
Check the performance criterion $ \mathcal{L}_2$-norm of $e^c$
\begin{alignat*}{3}
\int_0^\infty \dot{V}(e^c, \bm{\theta}) d\tau &= V(\infty) - V(0)\\
-| a_m + l |\int_0^\infty {e^c}^2 d\tau &= V(\infty) - V(0)\\
V(0) &= \underbrace{V(\infty)}_{\geq_0} + |a_m + l| \cdot \| e^c \|_2^2\\% \int_0^\infty (e^c)^2 d\tau\\
V(0) &\geq |a_m + l| \cdot \| e^c \|_2^2 \\
\| e^c \|_2 &\leq \sqrt{\dfrac{V(0)}{|a_m + l|}}
\end{alignat*}

\begin{conclusion}{darkgreen}
\begin{alignat}{3}
\| e^c \|_2^2 &\leq \frac{1}{2} \dfrac{(e^c(0))^2 + \frac{|k_p|}{\gamma} \bm{\theta}^T(0) \bm{\theta}(0)}{|a_m + l|} 
    \label{eq:ec-2norm}
\end{alignat}
\end{conclusion}

\paragraph{Discussion}
\begin{itemize}
\item Increasing $\gamma$ reduces $\| e^c \|_{\mathcal{L}_2}$
    depending on the parameter errors $ \tilde{\bm{\theta}}$ 
\item Increasing the value of $l$ reduces $\| e^c \|_{\mathcal{L}_2}$\\
    also from $e^c(0)$
\end{itemize}

\subsection{Analysing the signal oscillations}
$\mathcal{L}_2$-norm of $\dot{k}$
\begin{alignat*}{3}
\dot{k} &= -\gamma \sign{k_p} e^c r(t)\\
\int_0^\infty |\dot{k}|^2 d\tau &= \gamma^2 \int_0^\infty (e^c)^2r^2 d\tau\\
    & \qquad \left( r(t) \leq \| r \|_{\mathcal{L}_\infty} \right)\\
    &\leq \gamma^2 \| r \|^2_{\mathcal{L}_\infty} \int_0^\infty (e^c)^2 d\tau\\
    &\leq \gamma^2 \| r \|^2_{\mathcal{L}_\infty} \| e^c \|^2_{\mathcal{L}_2}\\
\end{alignat*}

\begin{conclusion}{darkgreen}
\begin{alignat*}{3}
\| \dot{k} \|_2 &\leq \gamma \| r \|_{\mathcal{L}_\infty}
    \sqrt{\dfrac{V(0)}{|a_m + l|}}
\end{alignat*}
\end{conclusion}

Increasing $\gamma$ or reducing $l$ causes $\dot{k}$ to decrease in magnitude.\\

$\mathcal{L}_2$-norm of $\dot{\theta}$
\begin{alignat*}{3}
\dot{\theta} &= -\gamma \sign{k_p} e^c x_p(t)\\
    &= -\gamma \sign{k_p} e^c \left( e^c + x_m(t) \right)\\
|\dot{\theta}|^2 &= \gamma^2 (e^c)^2 \left( e^c + x_m(t) \right)^2\\
    & \qquad \qquad  \measuredVar{ (a+b)^2 \leq 2a^2 + 2b^2} \\
    &\leq 2 \gamma^2 (e^c)^2 \left[ (e^c)^2 (e^c)^2 + x_m^2 \right]\\
\int_0^\infty |\dot{\theta}|^2 d\tau
    &\leq 2\gamma^2 \left[ \int_0^\infty (e^c)^2 (e^c)^2 d\tau +
        \int_0^\infty (e^c)^2 x_m^2 d\tau \right]\\
    &\vdots\\
\end{alignat*}

\begin{conclusion}{darkgreen}
\begin{alignat*}{3}
    & \quad  \int_0^\infty |\dot{\theta}|^2 d\tau  \\
&\leq 2\gamma^2 \frac{V(0)}{|a_m+l|}
    \left[ V(0) \left( 2 + \frac{l^2}{|a_m| \cdot |a_m + l|} \right) \right. \\
         & \qquad \qquad \qquad \qquad 
         + 2 \| \dot{x}_m \|^2_{\mathcal{L}_\infty} \biggr]
\end{alignat*}
\end{conclusion}

\paragraph{Discussion}
\begin{itemize}
\item $l$ reduces contribution of the ORM
    $\| \dot{x}_m(t) \|_{\mathcal{L}_\infty}$ on $\| \dot{\theta} \|_{\mathcal{L}_2}$ 
\item $l$ has no clear effect on the contributions of $V(0)$.
\item $\gamma$ always increases the oscillations, s.
    $\| \dot{\theta} \|_{\mathcal{L}_2}$
\end{itemize}






% % \subsection{Lemma: ``Adaptive laws based on Lyapunov for
% % relative degree 1 plants''}
% % 
% % Consider the dynamical system below%
% % \footnote{think of the error dynamics, not of the plant!}.
% % \begin{alignat*}{3}
% % \dot{x}(t) &= A x(t) + b \theta^T(t) \phi(t) \quad & x &\in \R^n\\
% % y(t) &= c^Tx(t) \quad           & y, z_1 &\in \R^1\\
% % z_1(t) &= ky(t)             & \phi, \theta & \in \R^k
% % \end{alignat*}
% % ``$z_1$ allows change of symbol.\\
% % 
% % where $(A,b)$ is stabilisable and $(c^T,A)$ 
% % is detectable and $c^T(sI-A)^{-1}b \corresponds H(s)$ 
% % is SPR.\\
% %  
% % Let $\theta(t)$ be a vector of adjustable parameters.\\
% % Let $\phi(t)$ and $z_1(t)$ be time-varying functions that can
% % be measured.\\
% % 
% % Then, if $\theta(t)$ is adjusted as
% % \begin{alignat}{3}
% % \dot{\theta}(t) &= -\sign{k} z_1(t) \phi(t)
% % \end{alignat}
% % (Adaptive law for multistate systems)
% % 
% % $\Rightarrow$ the equilibrium state $(x=0, \theta=0)$ 
% % is uniformly stable in the large.\\
% % uniformly stable: not dependent on time nor on IC\\
% % in the large: IC don't matter everywhere in $\R^n$.
% % 
% % \paragraph{Proof}
% % Since $H(s)$ is SPR, it follows from KY lemma that $\exists P = P^T >0$ 
% % such that
% % \begin{alignat}{3}
% % A^TP + PA = -Q, \quad \txt{with } Q^T = Q>0\\
% % Pb &= c
% % \end{alignat}
% % 
% % Let $V$ be a positive definite function
% % \begin{alignat}{3}
% % V &= x^TPx + \frac{1}{|k|} \theta^T \theta
% % \end{alignat}
% % 
% % Its time derivative along the system is
% % \begin{alignat}{3}
% % \dot{V} &= x^T \left( PA + A^TP \right)x + 2 x^TPb \theta^T\phi - 2\theta^Ty\phi\\
% %         &= -x^TQx \leq 0
% % \end{alignat}
% % 
% % Therefore, the origin of (1), (2) is stable.
% % %% TODO correct references.
% % 
% % \paragraph{Discussion}
% % We now have a simple tool for finding adaptive laws with
% % error dynamics
% % \begin{alignat*}{3}
% % e &= \frac{1}{k^*} M(s) [ \tilde{\theta}^T \phi]
% % \end{alignat*}
% % (where $M(s)$ SPR(s)tabilisable and detectable)
% % and the adaptive laws are
% % \begin{alignat}{3}
% % \dot{\theta} &= -\sign{\varepsilon} e \Gamma \phi
% % \end{alignat}
% % 
% % If $M(s)$ has relative degree 1, it is obvious that
% % $e(t) \rightarrow 0$ for $t \rightarrow \infty$.
% % (Why?)
% % 
% % \subsection{Performance considerations}
% % Engineering criteria:\\
% % \begin{tabu} to \columnwidth{|c|c|}
% % \toprule
% % Performance & Noise\\
% % \midrule
% % Disturbance & Robustness
% % \bottomrule
% % \end{tabu}
% % 
% % \begin{itemize}
% % \item Increasing $\gamma$, we are unhappy with the oscillations of our 
% %     parameters $\theta(t)$ and therefore with the
% %     oscillations of $u(t)$.
% % \item We have no clue what the adaptive closed loop will do between
% %     $t=0$ and $t=\infty$ other than boundedness.
% % \end{itemize}
% % 
% % \paragraph{Now}
% % Deal with transient response.
% % 
% % \paragraph{Idea}
% % Adaptation changes with signals.
% % \begin{alignat*}{3}
% % \left( \dot{\theta}(t) = - \sign{\varepsilon} e \gamma \phi \right),
% % \quad e, \gamma \txt{ changeable}
% % \end{alignat*}
% % 
% % We can alter the transient with $\gamma$ (leads to oscillations),
% % or we can change $e(t)$.
% % 
% % \paragraph{So far} Open loop reference model (ORM)
% % \begin{alignat}{3}
% % \dot{x}_m(t)^o &= a_m x_m(t)^o + k_m r(t)\\
% % \end{alignat}
% % 
% % %% TODO control block
% % 
% % \paragraph{Now} Closed loop reference model (CRM)
% % \begin{alignat}{3}
% % \dot{x}_m(t)^c &= a_m x_m(t)^c + k_m r(t) - l e^c(t)
% % \end{alignat}
% % 
% % ``CRM is observer-like; $M$ helps $G$ by moving towards
% % $G$ and retreating to original position.''
% % 
% % ORM $=$ CRM if $l=0$.
