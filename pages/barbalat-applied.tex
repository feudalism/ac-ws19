Using Barbalat's Lemma (Variant A)%
\footnote{Chapter \vref{ch:barbalat}}
 on the function $V$,
we need to fulfill the following conditions:
\begin{enumerate}[label=(\roman*)]
\item $V$ is differentiable\\
    \greentext{Yes, $\exists~ \dot{V} = a_m e^2(t)$}
\item $V$ has a finite limit as $t \rightarrow \infty$ \\
    \greentext{Yes, as $V \succeq 0$ and  $\dot{V} \preceq 0$}
\item $\dot{V}$ is uniformly continuous $\Leftarrow \exists ~ \ddot{V}$
    (sufficient condition%
    \footnote{Chapter \vref{ch:uniform-continuous}}    
    )\\
    \redtext{Is $\ddot{V} = 2 a_m e \dot{e}$ bounded?}
\end{enumerate}~

Boundedness of $e(t)$
\begin{itemize}
\item As $V$ is bounded from below and non-increasing,
    $V$ has a limit as $t \rightarrow \infty$.
\item Tracking error $e(t)$ and parameter errors $ \tilde{\bm{\theta}}(t)$
    are bounded.
\item As $\tilde{\bm{\theta}}(t)$ bounded and $\bm{\theta}^*$ constant,
    $\bm{\theta}(t)$ is bounded.
\end{itemize}~

Boundedness of $\dot{e}(t)$ 
\begin{itemize}
\item \redtext{Assume $r(t)$ bounded}%
    \footnote{Reasonable assumption, because why would we want
    to use an unbounded input?}, then, from the reference
    model equation \eqn{eq:M}:\\
    $x_m(t), \dot{x}_m(t)$ bounded ($\because~ M$ is stable)
\item $x_p(t) = \bounded{e(t)} + \bounded{x_m(t)}$\\
    $\Rightarrow x_p(t)$ bounded.
\item $u(t) = \bm{\theta}^T(t) \bm{\phi}(t)$ bounded if $\bm{\phi}(t)$ bounded.\\
    $\bm{\phi}(t) = \matr{\bounded{r(t)} & \bounded{x_p(t)} & \bounded{f(z)}}^T$\\
    (new requirement: \redtext{$f(z)$ needs to be bounded}).
\item $\dot{x}_p(t) = a_p \bounded{x_p(t)} + k_p \bounded{u(t)}$\\
    $\Rightarrow \dot{x}_p(t)$ bounded
\item $\dot{e}(t) = \bounded{\dot{x}_p(t)} - \bounded{\dot{x}_m(t)}$
    is bounded.
\end{itemize}~

All the conditions of Barbalat's lemma thus fulfilled,
we can conclude that the derivative of $V$ approaches zero
for $t \rightarrow \infty$.
\begin{alignat*}{3}
& ~ & \lim_{t \rightarrow \infty} \dot{V} &= 0\\
\Rightarrow& ~ & \lim_{t \rightarrow \infty} e(t) &= 0
\end{alignat*}

