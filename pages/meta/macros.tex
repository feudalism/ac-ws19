%% COMMANDS - REGION
%\newcommand{\setdeflang}{\selectlanguage{ngerman}}

%% COMMANDS - graphics
% #1: file name without extension
% #2: width in percentage of page text width
% #3: style options, e.g. \normalsize
\newcommand{\inkscape}[3][1]{
        \def\svgwidth{#3\columnwidth}
        { \footnotesize #1
        \input{./img/#2.pdf_tex}
        }
    }
    
%% COMMANDS - Formatting
\newcommand{\hsc}[1]{{\small\MakeUppercase{#1}}}
\newcommand{\eqn}[1]{(\ref{#1})}
%% COMMANDS - Typesetting
\newcommand{\bluetext}[1]{{\color{blue}(#1)}}
\newcommand{\txt}[1]{\textnormal{#1}}
\newcommand{\tund}[2]{{#1}_{\txt{#2}}}
\newcommand{\overtext}[2]{\overset{\txt{#1}}{#2}}
\newcommand{\undertxt}[2]{\underbrace{#1}_{\textnormal{#2}}}
\newcommand{\code}[1]{\texttt{#1}}
\newcommand{\txtcirc}[1]{\textcircled{\footnotesize \raisebox{-.9pt} #1}}
\newcommand{\known}[1]{\textcolor{colKnown}{#1}}
\newcommand{\unknown}[1]{\textcolor{red}{#1}}
\newcommand{\bounded}[1]{\underset{\textnormal{\color{blue}bd.}}{#1}}

%% MATHEMATICAL OPERATORS
\newcommand{\set}[1]{\ensuremath{\left\lbrace #1 \right\rbrace}}
\newcommand{\paren}[1]{\ensuremath{\left( #1 \right)}}
\newcommand{\squareb}[1]{\ensuremath{\left[ #1 \right]}}
\newcommand{\matr}[1]{\begin{bmatrix}#1\end{bmatrix}}
\newcommand{\modl}[1]{\ensuremath{\left\lvert #1 \right\rvert}}
\newcommand{\norm}[1]{\ensuremath{\left\lVert #1 \right\rVert}}
\newcommand{\inv}[1]{\ensuremath{#1^{-1}}}
\newcommand{\quadr}[2]{\ensuremath{#1^T #2 #1}}
\newcommand{\pdein}[2]{\dfrac{\partial #1}{\partial #2}}
\newcommand{\pde}[1]{\dfrac{\partial}{\partial #1}}
\newcommand{\evalat}[2]{\left. #1 \right\rvert_{#2}}
\DeclareMathOperator*{\argmax}{arg\,max}
\DeclareMathOperator*{\argmin}{arg\,min}
\DeclareMathOperator*{\diag}{diag}
\DeclareMathOperator{\spn}{span}
\DeclareMathOperator{\sign}{sign}
\newcommand{\sgn}[1]{\sign \left( #1 \right)}
% Integral over time int{lowerbound}{upperbound}{inttegral}
\newcommand{\intdt}[3]{\ensuremath{\int_{#1}^{#2} #3 ~dt}}
% Sum over k {lowerbound}{upperbound}
\newcommand{\sumk}[2]{ \sum_{#1}^{#2} }


%% SYMBOLS
\def\lt{\ensuremath{\leftarrow~}}
\def\lat{\ensuremath{\rightarrow~}}
\def\inc{\ensuremath{\uparrow}}
\def\dec{\ensuremath{\downarrow}}
\def\R{\mathbb{R}}
\def\err{\mathcal{O}}
\def\mult{\ensuremath{\times}}
      
                     
%% VALUES


%% FORMULAE
