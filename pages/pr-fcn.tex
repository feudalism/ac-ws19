\begin{definition}{Positive real function I}
A rational function $H(s): \Z \rightarrow \Z$, $s = \sigma + j\omega$ is
positive real (PR), if
\begin{enumerate}[label=(\roman*)]
\item \legendsquare{babypink} $H(s)$ is real for real $s$ 
\item \legendsquare{babyblue} $\Re\{H(s)\} \geq 0$ for  $\Re\{s\} > 0$
\end{enumerate}

\begin{figure}[H]
    \centering
    \inkscape[\footnotesize]{pr-fcn}{0.81}
    \caption{Positive real mapping $s \rightarrow H(s)$}
    \label{fig:positive-real-mapping}
\end{figure}

\end{definition}

\begin{definition}{Positive real function II}
A rational function $H(s)$ is positive real, if
    \begin{enumerate}[label=(\roman*)]
    \item $H(s)$ is analytic
    %\footnote{\textbf{analytic}: ``converges to Taylor series, somewhat smooth.''}
     in $\real{s} >0$ 
        \begin{itemize}
        \item $H(s)$ has no poles in RHP ($\real{s} > 0$)
        \item $H(s)$ is stable
        \end{itemize}
    \item $\real{H(j\omega)} \geq 0, \quad \forall\omega \in [0, \infty]$    
        \begin{itemize}
            \item Nyquist of $H(s)$ is in the RHP
            \item phase $\angle H(j\omega) \in [-90^{\circ}, +90^{\circ}]$
            \item rel. degree of $H(s)$ is 0 or 1
        \end{itemize}
    \item any pure imaginary pole $j\omega$ of $H(s)$ is
        a simple pole, and the residue
    \begin{alignat*}{3}
    \lim_{s \rightarrow j\omega} \left( s - j\omega \right) H(s)
    \end{alignat*}
        is positive semidefinite.\\

        Alternatively: $H(\infty) > 0$ or 
    \begin{alignat*}{3}
        \lim_{\omega \rightarrow \infty} \omega^2 \real{H(j\omega)} \geq 0
    \end{alignat*}
    \end{enumerate}
\end{definition}~

\begin{definition}{Strictly positive real functions}
$H(s)$ is strictly positive real (SPR) if $H(s - \varepsilon)$ 
is PR for some $\varepsilon > 0$.
\end{definition}~

\paragraph{Note} relative degree of a system corresponds to
its response delay.

\begin{lemma}{SPR lemma}
$H(s)$ is SPR if
\begin{enumerate}[label=(\roman*)]
    \item $H(s)$ is Hurwitz\\
        all poles on LHP, none are purely imaginary
    \item $\real{H(j\omega)} > 0, \quad \forall \omega \in \R$
        \begin{itemize}
            \item Nyquist of $H(s)$ is in the RHP and \redtext{not on the imaginary axis}.
            \item phase $\angle H(j\omega) \in (-90^{\circ}, +90^{\circ})$
            \item rel. degree of $H(s)$ $\in \set{0,1}$
        \end{itemize}
        phase $\in (-90 ^{\circ}, 90 ^{\circ})$\\
        rel. degree $\in \set{0, 1}$
    \item $H( \infty) > 0$ (positive gain for proper $H$) or 
        \begin{alignat*}{3}
        \lim_{\omega \rightarrow \infty} \omega^2 \real{H(j\omega)} > 0
        \end{alignat*}
        positive gain for relative degree 1
\end{enumerate}
\end{lemma}

\paragraph{Discussion}
If $H(s)$ is SPR, then so is the inverse $H^{-1}(s)$
(stable poles, stable zeroes).\\

\begin{conclusion}{darkgreen}
    SPR  $\Rightarrow H(s)$ is stable, minimal-phase.\\
    I.e., only stable zeroes, because zeroes are in LHP.
    \begin{itemize}
        \item phase $\angle H(j\omega) \in (-90^{\circ}, +90^{\circ})$
        \item rel. degree of $H(s) \leq 1 (-1, 0, 1)$
        \item positive gain $\forall ~ \omega$
    \end{itemize}
\end{conclusion}~

\begin{example}{PR}
$G(s) = \frac{1}{s}$ has a single pole $s = 0$,
with a residue of 1.
\begin{alignat*}{3}
\real{G(j\omega)} = \real{ \frac{1}{j\omega}} = 0 \qquad \forall \omega \neq 0
\end{alignat*}

Hence, $G(s)$ is PR but not SPR,
as $ \frac{1}{s - \varepsilon}$ has a pole
in $\real{s} \geq 0$ for any
$\varepsilon > 0$. 
\end{example}

\begin{example}{PR}
$G(s) = \dfrac{1}{s + a}, a>0$ is Hurwitz.
\begin{alignat*}{3}
\real{G(j\omega)} &= \dfrac{a}{\omega^2 + a^2} > 0 \\
    & \qquad
    \forall \omega \in [0, \infty]\\
\lim_{\omega \rightarrow \infty} \omega^2 \real{G(j\omega)}
    &= \lim_{\omega \rightarrow \infty} \dfrac{\omega^2a}{\omega^2 + a^2}\\
    &= a, \qquad a > 0
\end{alignat*}
\end{example}

\begin{example}{PR}
\begin{alignat*}{3}
G(s) &= \dfrac{1}{s^2 + s + 1}\\
\real{G(j\omega)} &= \dfrac{1 - \omega^2}{(1 - \omega^2)^2 + \omega^2}\\
    &\ngtr 0 \qquad \forall \omega\\
    &\Rightarrow G \txt{ is not PR.}
\end{alignat*}
\end{example}


