\begin{definition}{Positive real function 1}
A rational function $H(s): \Z \rightarrow \Z$, $s = \sigma + j\omega$ is
positive real (PR), if
\begin{enumerate}[label=(\roman*)]
\item $H(s)$ is real for real $s$
\item  $\Re\{H(s)\} \geq 0$ for  $\Re\{s\} > 0$
\end{enumerate}
\end{definition}~

\begin{definition}{Positive real function 2}
A rational function $H(s)$ is positive real, if
\begin{enumerate}[label=(\roman*)]
\item $H(s)$ is analytic
%\footnote{\textbf{analytic}: ``converges to Taylor series, somewhat smooth.''}
 in $\Re\{s\} >0$ 
    \begin{itemize}
    \item $H(s)$ has no poles in RHP ($\Re \set{s} > 0$)
    \item $H(s)$ is stable
    \end{itemize}
\item $\Re \set{H(j\omega)} \geq 0, \quad \forall\omega \in [0, \infty]$    
    \begin{itemize}
        \item Nyquist of $H(s)$ is in the RHP
        \item phase $\angle H(j\omega) \in [-90^{\circ}, +90^{\circ}]$
        \item rel. degree of $H(s)$ is 0 or 1
    \end{itemize}
\item any pure imaginary pole $j\omega$ of $H(s)$ is
    a simple pole, and the residue
\begin{alignat*}{3}
\lim_{s \rightarrow j\omega} \left( s - j\omega \right) H(s)
\end{alignat*}
    is positive semidefinite.\\

    Alternatively: $H(\infty) > 0$ or 
 \begin{alignat*}{3}
\lim_{\omega \rightarrow \infty} \omega^2 \Re \set{H(j\omega)} \geq 0
\end{alignat*}
\end{enumerate}
\end{definition}~

\begin{definition}{Strictly positive real functions}
$H(s)$ is strictly positive real (SPR) if $H(s - \varepsilon)$ 
is PR for some $\varepsilon > 0$.
\end{definition}~

\begin{definition}{SPR lemma}
$H(s)$ is SPR if
\begin{enumerate}[label=(\roman*)]
\item $H(s)$ is Hurwitz\\
    all poles on LHP, none are purely imaginary
\item $\Re \set{H(j\omega)} > 0, \quad \forall \omega \in \R$\\
    phase $\in (-90 ^{\circ}, 90 ^{\circ})$\\
    rel. degree $\in \set{0, 1}$
\item $H( \infty) > 0$ (positive gain for proper $H$) or 
\begin{alignat*}{3}
\lim_{\omega \rightarrow \infty} \omega^2 \Re \set{H(j\omega)} > 0
\end{alignat*}
positive gain for relative degree 1
\end{enumerate}
\end{definition}

\paragraph{Note} relative degree of a system corresponds to
its response delay.

\paragraph{Discussion}
If $H(s)$ is SPR, then so is the inverse $H^{-1}(s)$
(stable poles, stable zeroes).\\

SPR  $\rightarrow H(s)$ is stable, minimal-phase.
I.e., only stable zerioes, because zeroes are in LHP.
\begin{alignat*}{3}
\txt{rel. degr.} &\leq 1 \qquad(-1, 0, 1)\\
\angle H(j\omega) &\in (-90 ^{\circ}, 90 ^{\circ})\\
     & \quad \txt{ with positive gain } \forall \omega
\end{alignat*}
