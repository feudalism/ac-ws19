\subsection{Uniformly continuous functions}
\label{ch:uniform-continuous}
\begin{definition}{Uniformly continuous function}
A function $f(t): \R \rightarrow \R$ 
is uniformly continous, if\\
$\forall \varepsilon>0:
    \quad \exists~ \delta = \delta(\varepsilon) >0,$
\begin{alignat*}{3}
\forall |t_2 - t_1| &\leq \delta\\
\Rightarrow |  f(t_2) - f(t_1) | &\leq \varepsilon
\end{alignat*}
%\label{ch:uniform-continuous}
\end{definition}

\begin{conclusion}{darkgreen}
\textbf{Sufficient condition for uniformly continuous functions}:
If the derivative $\dot{f}(t)$ exists (i.e. bounded),
$\Rightarrow f(t)$ is uniformly constant.
\end{conclusion}~

\subsection{Barbalat's lemma}
\label{ch:barbalat}
\begin{lemma}{Barbalat Variant A}
If $f(t): \R \rightarrow \R$
\begin{enumerate}
\item is a differentiable function
\item has a finite limit as $t \rightarrow \infty$ 
\item $\dot{f}(t)$ is uniformly continuous
\end{enumerate}
$\Rightarrow \lim_{t \rightarrow \infty} \dot{f}(t) = 0$\\
\end{lemma}

\begin{lemma}{Barbalat Variant B}
If
\begin{enumerate}
\item $f(t): \R \rightarrow \R$ is uniformly continous $\forall t$
\item $\exists~ \lim_{t \rightarrow \infty} \int_0^t f(\tau) d\tau$
\end{enumerate}
$\Rightarrow \lim_{t \rightarrow \infty} f(t) = 0$\\
\end{lemma}

\begin{lemma}{Barbalat Variant C}
If $f, \dot{f} \in \mathcal{L}_\infty$ 
and $f \in \mathcal{L}_2$,
then  $|f(t)| \rightarrow 0$ as $t \rightarrow \infty$.
\end{lemma}


\begin{definition}{Signal norm}
\paragraph{Idea} quantify magnitude of a signal $x(t)$
-- ``How big is a signal?'' -- also called the $\mathcal{L}_p$ space.\\

\paragraph{Given}
\begin{alignat*}{3}
x(t): \R^+ \rightarrow \R^n, \quad  \quad \R^+ = [0, \infty)
\end{alignat*}

\paragraph{p-Norm}
    \begin{alignat*}{3}
    \|x_p\| &= \left( \int_0^\infty |x(t)|^p dt \right)^{1/p}
        & \qquad p \in (0, \infty)
    \end{alignat*}
\end{definition}~

\begin{itemize}
\item If $\bm{x}(t)$ vector, $|.|$ is the vector 2-norm,
    `\textbf{distance}'.
\item $\|x\|_\infty = \sup_{t \in \R^+} |x(t)| \corresponds$ 
    highest value of $\bm{x}$\\
    ``When power $\infty$, only the greatest value survives''
\end{itemize}~

\begin{definition}{Functional space}
\begin{alignat*}{3}
\mathcal{L}_p = \{x(t) \in \R^n : \undertxt{\| x \|_p < \infty}{exists} \}
\end{alignat*}
\end{definition}~

\begin{example}{Functional space}
\begin{itemize}
\item $x(t) \in \mathcal{L}_p$: $x$ is bounded.\\
``$x$'s highest value exists and is not infinity.''
\item Show that $e \in \mathcal{L}_\infty$ \\
$e$ is in $V$ and $V$ is bounded, $\therefore e \in \mathcal{L}_\infty$.
\item Show that $e \in \mathcal{L}_2$ 
\begin{alignat*}{3}
\int_0^\infty \dot{V} dt &= V(\infty) - V(0), \qquad \txt{is bounded}.\\
a_m \int_0^\infty e^2 dt &\in \mathcal{L}_\infty\\
\Rightarrow e &\in \mathcal{L}_2
\end{alignat*}
\end{itemize}
\end{example}



%% TODO FILL IN

