\begin{alignat*}{3}
\dot{x}_m^c(t) &= a_m x_m^c(t) + k_m r(t) - le^c(t)
    \tag{\ref{eq:M-crm}}\\
\dot{x}_p(t) &= a_p x_p(t) + k_p u(t) 
    \tag{\ref{eq:G-linear}}
\end{alignat*}

Input is
\begin{alignat*}{3}
u &= \matr{a(t) & k(t)} \matr{x_p(t) \\ r(t)}% =  a(t) x_p(t) + k(t) r(t)\\
    = \bm{\theta}^T(t) \bm{\phi}(t) 
\end{alignat*}

\begin{alignat*}{3}
\dot{x}_p(t) &= a_m x_p(t) + k_m r(t) + k_p \bm{\theta}^T(t) \bm{\phi}(t)
\end{alignat*}

Tracking error
\begin{alignat*}{3}
\dot{e}^c(t) &= \dot{x}_p(t) - \dot{x}_m^c(t)\\
    &= \left( a_m + l \right) e^c(t) + k_p \bm{\theta}^T(t) \bm{\phi}(t)
\end{alignat*}

Lyapunov-like function
\begin{alignat*}{3}
V(e^c, \tilde{\bm{\theta}})
    &= \frac{1}{2} (e^c)^2 + \frac{1}{2} \Gamma^{-1} |k_p| \bm{\tilde{\theta}}^T(t) \bm{ \tilde{\theta}}(t)\\
\dot{V}
    &= e^c \dot{e}^c + \Gamma^{-1} |k_p| \tilde{\bm{\theta}}^T \bm{\dot{ \tilde{\theta}}} = \dots\\
    &= \left( a_m + l \right) (e^c)^2\\
    & \quad     + \underbrace{e^c k_p \tilde{\bm{\theta}} \bm{ \phi}
        + \Gamma^{-1} |k_p| \tilde{\bm{\theta}}^T \bm{\dot{ \tilde{\theta}}}}_{ \overset{!}{=} 0}\\
\dot{V} &= \left( a_m + l \right) (e^c)^2 \leq 0, \qquad l < 0
\end{alignat*}

Adaptive law
\begin{alignat*}{3}
\dot{\bm{\theta}} &= -\Gamma \sign{k_p} e^c \bm{\phi}
\end{alignat*}

\paragraph{Proof} as before.
$e^c(t) \rightarrow 0$ for $t \rightarrow \infty$ %
\footnote{We assume here that $e^c(t) \rightarrow 0$ follows from
$e^o(t) \rightarrow 0$. In actuality, though, $e^o(t)$ can't
be proven for special functions. However, these cases
are usually not relevant to engineering/industry.
Therefore, \redtext{strictly speaking}, we can't actually assume
that $e^c(t) \rightarrow 0$}.

\paragraph{Questions}
\begin{itemize}
\item How do we show increased performance?\\
    (using $\| e^c(t) \|_{ \mathcal{L}_2}$ as a performance criterion)
\item How do we show that the oscillations decrease?
\end{itemize}

