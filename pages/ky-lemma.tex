``Maier version; how to design a controller given SPR''\\

\begin{lemma}{Kalman-Yakobovich Lemma}
    \paragraph{Given}
    \begin{itemize}
    \item a scalar $\gamma \geq 0$
    \item vectors $\bm{b}$ and $\bm{c}$,
    \item an asymptotically stable matrix $\bm{A}$%
        \footnote{positive eigenvalues}, and
    \item a positive definite matrix $\bm{L} \succ 0$,
    \end{itemize}
    \begin{alignat*}{3}
    \txt{If } H(s) &\corresponds \frac{1}{2} \gamma + \bm{c}^T (s\bm{I}-\bm{A})^{-1} \bm{b}\\
        \Rightarrow H(s) &\txt{ is SPR}
    \end{alignat*}~

    Then, there exist
    \begin{itemize}
    \item a scalar $\varepsilon >0$
    \item a vector $\bm{q}$, and
    \item a symmetric positive definite matrix $\bm{P}$,
    \end{itemize}
    s.t.
    \begin{alignat}{3}
    \bm{A^TP + PA}   &= -\bm{qq}^T - \varepsilon \bm{L}
        \label{eq:ky-lemma-atp}\\
    \bm{Pb} - \bm{c}      &= \sqrt{\gamma} \bm{q}
        \label{eq:ky-lemma-bc}
    \end{alignat}
\end{lemma}

\paragraph{Using it}
We only need $\gamma=0$ in all cases (in this course).
Hence we can say: if $H(s)$ is SPR $\Rightarrow \exists ~ \bm{P}=\bm{P}^T>0$,\\
s.t.
\begin{alignat*}{3}
\bm{A^TP + PA}   &= -\bm{Q}\\
\bm{Pb} &= \bm{c} \qquad \txt{(s. note)}\footnotemark  \\
\end{alignat*}%
\footnotetext{"boundary cond., means SPR"}%
where $\bm{Q}=\bm{Q}^T>0$

\begin{lemma}{Adaptive laws based on Lyapunov}
(For rel. degree 1 plants)\\
Consider the dynamical system below%
\footnote{This refers to the error dynamics, \redtext{not the plant}!}%
\footnote{$z_1$ allows change of symbol with respect to the output $y(t)$}%
\footnote{$x \in \R^n,
        \qquad y, z \in \R^1,
        \qquad \bm{\phi}, \bm{\theta} \in \R^k$}.
\begin{alignat*}{3}
\dot{\bm{x}}(t) &= \bm{A} \bm{x}(t) + \bm{b} \bm{\theta}^T(t) \bm{\phi}(t)
    \numberthis \label{eq:G-mimo}\\
y(t) &= \bm{c}^T \bm{x}(t)     \\
z_1(t) &= k y(t) 
\end{alignat*}

where
\begin{itemize}
\item $(\bm{A}, \bm{b})$ is stabilisable
\item $(\bm{c}^T, \bm{A})$ is detectable
\item $\bm{c}^T \left( s \bm{I} - \bm{A} \right)^{-1} \bm{b} \corresponds H(s)$
    is SPR
\end{itemize}~

Let $\bm{\theta}(t)$ be a vector of adjustable parameters.\\
Let $\bm{\phi}(t)$ and $z_1(t)$ be time-varying functions that can be measured.\\

Then, if $\bm{\theta}(t)$ is adjusted as
\begin{alignat}{3}
\bm{\dot{\theta}}(t) &= - \sign{k} z_1(t) \bm{\phi}(t)
\label{eq:adaptive-law-multistate}
\end{alignat}

$\Rightarrow$ the \greentext{equilibrium state} $\left( \bm{x}=0, \bm{\theta}=0 \right)$ 
is \greentext{uniformly stable%
\footnote{uniformly stable: not dependent on time nor on IC}
 in the large%
\footnote{in the large: IC don't matter anywhere in $\R^n$}}.
\end{lemma}

\paragraph{Proof}
Since $H(s)$ is SPR, it follows from
the KY-lemma that $\exists ~ \bm{P} = \bm{P}^T > 0$,
such that
\begin{alignat*}{3}
\bm{A^TP + PA}   &= -\bm{Q}, \quad \bm{Q} = \bm{Q}^T > 0
    \tag{\ref{eq:ky-lemma-atp}}\\
\bm{Pb} &= \bm{c} 
    \tag{\ref{eq:ky-lemma-bc}}
\end{alignat*}%

Let $V$ be a positive definite function
\begin{alignat*}{3}
V &= \bm{x}^T \bm{P} \bm{x} + \dfrac{1}{|k|} \bm{\theta}^T \bm{\theta}\\
\dot{V} &= \bm{x}^T \left(  \bm{PA + A^TP} \right) \bm{x}\\
        &\quad + 2 \bm{x^T Pb \theta^T \phi} - 2 \bm{\theta^T} y \bm{\phi}\\
    &= - \bm{x^TQx} \leq 0
\end{alignat*}

Therefore, the origin of the system \eqn{eq:G-mimo}
together with the adaptive law \eqn{eq:adaptive-law-multistate}
is stable.

\paragraph{Discussion}
We now have a simple tool for finding adaptive laws
with error dynamics below \eqn{eq:error-dynamics-nonlinear}, where $M(s)$ is SPR,
stabilisable and detectable,
and the adaptive laws are defined as in \eqn{eq:adaptive-law-eps}
\begin{alignat*}{3}
e(t) &= \dfrac{1}{ \unknownVar{k^*}} \knownVar{M(s)} \matr{ \unknownVar{ \tilde{\bm{\theta}}^T \bm{\phi}}}
    \tag{\ref{eq:error-dynamics-nonlinear}}
\dot{\bm{\theta}}(t) &= - \sign{\varepsilon} e \bm{\Gamma} \bm{\phi}
    \numberthis \label{eq:adaptive-law-eps}
\end{alignat*}

If $M(s)$ has a relative degree of 1, it is obvious that $e(t) \rightarrow 0$
for $t \rightarrow \infty$%
\footnote{Why?}.
